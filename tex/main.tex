\documentclass{article}
\usepackage{natbib}
\bibpunct{(}{)}{;}{a}{,}{,}
\bibliographystyle{abbrvnat}

\usepackage[utf8]{inputenc}
\usepackage[T1]{fontenc} 
\usepackage{amsfonts,amssymb,amsthm}
\usepackage{amsmath,amscd}
\usepackage{algorithmicx,algorithm}
\usepackage[noend]{algpseudocode}
\usepackage{fullpage}
\usepackage{subcaption}
\usepackage{bm}
\usepackage{bbm}
\usepackage{centernot} 
\usepackage{enumerate} 
\usepackage{parskip}
\usepackage{pb-diagram} 
\usepackage{mathrsfs}
\usepackage[OT2,T1]{fontenc}
\usepackage{seqsplit}
\usepackage{enumitem}
\usepackage{color}
\usepackage{array}
\usepackage{verbatim}
\usepackage{url}
\usepackage{tikz, graphicx}
\usepackage{mathtools} %For mapsto
\usetikzlibrary{shapes, arrows, calc, positioning}
\usepackage{wrapfig, blindtext}
\usepackage{mleftright}
\usepackage{pgffor} %The for loop
%\usepackage{euler} %For the \mathscr
\usepackage{float} % For figure
%Bibtex
% \usepackage[nottoc]{tocbibind}


\newcommand{\todo}[1]{\textcolor{red}{\textbf{TODO:} #1}}
%encadre
\usepackage{tcolorbox}

\usepackage[colorlinks=true,
            linkcolor=blue,
            citecolor=blue,
            urlcolor=blue]{hyperref}
\usepackage{cleveref}



\title{Developing competences on theorical and applied aspects}
\author{Quoc-Bao Do \\
5th Year Student, Department of Applied Mathematics, INSA Toulouse, France \\
Mentor: Pierre Weiss \\
Researcher, Integrative Biology Center, Toulouse, France}
\date{January 2025}

\begin{document}

\maketitle

\tableofcontents

\newpage

\section*{Introduction}
In this report, I present my two recent projects that I had have contributed/worked on during my internship and my 5th year at INSA Toulouse. These two projects are under the supervision of my mentor and in collaboration with researchers from the Integrative Biology Center (CBI) in Toulouse, France. The first project consists of analytically analyzing the convolutional neural networks (CNNs) under the len of Minimum Mean Square Error (MMSE) estimator. We aim to gain insights into their performance, generalization capabilities, and their sensibility to various factors. This theoretical foundation will help inform the design and optimization of CNN architectures for various applications. The second project aims to develop a web-based application that utilizes deep learning techniques to detect and classify cyanobacterias from microscope images. While these two projects may seem distinct, they are done under the neccesity of advanced technologies that can solve real-world problems that the biologists and researchers from CBI are facing. In other hands, these two works form my set of skills both in theoretical and applied aspects, which are equivalently essential for my future career as a researcher in applied mathematics.

This report is structured as follows: Section~\ref{MMSE_analytical} presents the summary of the first project, including the mathematical formation, result, experimental validation and conclusion. Section~\ref{cyanoBacIdentification} introduces the web-based application for cyanobacteria detection, detailing the data acquisition process, deep learning approach, application deployment, preliminary results and conclusion. Finally, Section 4 concludes the report with a summary of key results and future directions for both projects.


\section{An analytical theory of convolutional neural network inverse problem solvers}\label{MMSE_analytical}
\noindent
\begin{minipage}[t]{0.48\linewidth}
\small
\vspace{0.4cm}
Convolutional Neural Networks (CNN) and other architectures~\citep{jin2017deep,mccann2017convolutional, zhu2018image} now achieve unprecedented performance in domains ranging from medical imaging to astrophysics and computer vision.
Yet, serious concerns have arisen regarding their reliability and robustness~\citep{antun2020instabilities}, with conflicting reports on whether small perturbations in the measurements or the forward operator lead to stable or unstable reconstructions~\citep{genzel2022solving}.
This work aims to provide a better understanding of the behavior of CNNs trained in a supervised manner for inverse problems. This is essential for defining the true domain of validity of such methods, assessing their stability, and enabling comparisons that go beyond empirical scoreboards. 
Our approach is based on the derivation of a closed-form MMSE estimator by incorporating functional constraints that capture two fundamental inductive biases of CNNs: translation equivariance and locality.
In the context of generative diffusion models, \citep{kamb2025an,scarvelisclosed} derive closed-form expressions for the MMSE under additive Gaussian noise (denoising) with architectural constraints such as equivariance and locality, and use them to study memorization and generalization. 
Locality has also been examined in~\citep{kadkhodaie2023learning} for modeling patch distributions.
For imaging inverse problems, equivariance has been widely explored in self-supervised learning~\citep{chen2023imaging,terris2024equivariant}.
In the supervised setting for linear inverse problems, it is standard to assume that a trained neural network approximates the MMSE~\citep{adler2018deep}. 
To the best of our knowledge, for general inverse problems, our work is the first to introduce additional functional constraints to more accurately characterize the action of neural networks and to derive closed-form expressions for the resulting constrained MMSE estimators. \todo{to reformulate}
\vspace{1cm}

My contribution to this work includes:
\begin{enumerate}
    \item Participating to derive theorical results
    \item implementing analytical expressions and optimizing numerical computations
    \item Training CNNs to validate to theorical results
    \item Participating to the writing of the scientific paper presenting this work.
\end{enumerate}

\end{minipage}
\hfill
\begin{minipage}[t]{0.48\linewidth}
\begin{figure}[H]
\centering
\includegraphics[
  trim=30mm 0 0 0,
  clip,
  width=\linewidth
]{./tex_figure/teasor_v2_standalone.pdf}
\caption{
Our analytic theory accurately predicts neural network outputs across settings. We consider three inverse problems: denoising, inpainting, and deconvolution (left to right), on FFHQ, CIFAR10, and FashionMNIST datasets (top to bottom) with varying noise levels $\sigma$ (columns). For each setting, we show the measurements (top row), our analytic LE-MMSE estimator (second row), and outputs of trained UNet, ResNet, and PatchMLP models (last three rows). Theory closely matches network outputs.
}
\label{fig:teaser}
\end{figure}
\end{minipage}



\subsection{Problem formulation}
 Let $\vx \in \mathbb{R}^{N}$ is a random vector following an distribution $\pdata$, $A: \mathbb{R}^{N}\to \mathbb{R}^M$ a linear or non linear operator, and $\ve \sim \Normal{0, \sigma^2 \Id_M}$ a Gaussian noise independent of $\vx$. The measurement $\vy \in \mathbb{R}^M$ is given by the following forward model:
\begin{equation}
    \label{eq:forward}
    \vy = A \vx 
\end{equation}
We aim to construct an estimator $\hat{x}(\vy)$ estimating $\vx$ given $\vy$.
A modern approach is to use neural networks to solve this inverse problem, some are physics-aware, \ie incorporate the approximate inverse of $A$ in the architecture, while others are physics-agnostic, \ie do not require  any prior knowledge of $A$.
In this work, we focus on linear operators $A$ in the numerical experiments for didactic purposes, but our theoretical results hold for nonlinear operators $A$ as well.
We consider estimators of the form:
\begin{equation} \label{eq:estimator_shape}
    \hat{x}(\vy) = \neural(B \vy),
\end{equation}
where $\neural: \R^N \to \R^N$ is a neural network with weights $w \in \Theta$ with $\Theta$ a set of admissible weights, and $B \in \R^{N \times M}$ is a pre-inverse operator that can be chosen freely, in particular, the neural network can be physics-agnostic, \ie $B = \Id$ if $M=N$,  or physics-aware if $B$ is an approximate inverse of $A$ such as the transpose $A^\top$, the Moore-Penrose pseudoinverse $A^+$, or a regularized inverse.
They can be considered as a robust baseline, and be interpreted as a single-step variant of unrolled neural networks~\citep{adler2018learned,Celledoni_2021_equivariant_neural_networks}, which often achieve top performance in empirical benchmarks~\citep{muckley2021results}.


\paragraph{Empirical MMSE estimators}
We study neural networks $\phi^\star_\M$ trained as follows.
\begin{graybox}
\begin{definition}[Constrained empirical MMSE estimators\label{def:constrained_mmse}]
    Let $\D$ be a dataset of clean images. 
    We define the empirical MMSE estimators $\hat x_{\M} \eqdef \phi^\star_{\M} \circ B$, where % $\phi^\star_\M$ is the solution of
\begin{equation}\label{eq:constrained_MMSE}
    \phi^\star_\M \eqdef \argmin_{\phi \in \M} \; \frac{1}{2} \Mean{\norm{\phi (B \vy) - \vx}^2}.
\end{equation}
Here, $\vx$ follows the empirical distribution of the dataset $p_{\D} \!=\! \frac{1}{|\D|} \sum_{x \in \D} \delta_{x}$ and $\vy$ defined by~(\ref{eq:forward}). 
The set $\M$ encodes the range of functions reachable by the neural network,  $\M \!=\! \{\neural : w \in \Theta\}$, with $\Theta$ a set of admissible weights.    
\end{definition}    
\end{graybox}
Explicitly characterizing the set $\M$ given the neural network architecture $\neural$ is mathematically intractable.
Instead, we approximate $\M$ by considering larger functional classes that capture the architectural constraints of $\neural$ as follows.
\begin{definition}[MMSE, E-MMSE, LE-MMSE]
	\label{def:mmse}
    Our estimators of interest are of the form $\hat x = \phi^\star_\M\circ B$, where $\phi^\star_\M$ is the solution of~(\ref{eq:constrained_MMSE}).
    We define in~\cref{tab:summarize_mmse_def} these estimators with various choices of $\M$. 
    \begin{center}
        \begin{table}[H]
            \centering
            \vskip -0.1in
            \caption{Different constraint sets correspond to different variants of the MMSE estimator.}\label{tab:summarize_mmse_def}
            \vskip -0.1in
            \begin{small}
            % \resizebox{\columnwidth}{!}{
                \setlength{\tabcolsep}{30pt}
                \begin{sc}
                    \begin{tabular}{llll}
                    \toprule
                    & Constraint $\M$ & Estimator & See\\
                    \midrule
                    (1) & Any measurable func. & MMSE -- $\xmmse$ &(\ref{prop:expression_MMSE}) \\
                    (2) & (1) + Translation equiv. & E-MMSE -- $\xtrans$ &(\ref{thm:equivariant_mmse}) \\
                        & (2) + Locality & LE-MMSE -- $\xtransloc$ &(\ref{theorem:local_trans_estimator}) \\
                    \bottomrule
                    \end{tabular}
                \end{sc}
            % }
        \end{small}
        \end{table}
        \vskip -0.4in
    \end{center}
\end{definition}%
These function classes capture differents architectural constraints of the neural networks $\neural$.
According to the universal approximation theorem~\citep{hornik1989multilayer}, a sufficiently deep and large Multi-Layer Perception (MLP) can approximate any measurable function, hence the MMSE  $\xmmse$ serves as the theoretical limit for large-scale MLPs. 
The E-MMSE $\xtrans$ approximates CNNs without constraints on the receptive field, while the LE-MMSE $\xtransloc$ approximates the set of functions reachable by CNNs with finite receptive fields. 

\subsection{Unconstrainted MMSE estimator}
We etablish the closed-form expression of the unconstrained MMSE estimator $\xmmse$.
\begin{graybox}
    \begin{proposition}[Closed-form of the MMSE \label{prop:expression_MMSE}]
	The MMSE estimator in \Cref{def:mmse} writes, for any $y \in \Y$,
    \begin{equation}
        \xmmse(y) = \sum_{x\in\D} x \cdot w(x \vert y), \label{eq:empirical_mmse}  \\
    \end{equation}
    where $w(x \vert y) = \frac{\Normal{B y; B A x, \sigma^2 B B^{\top}}}{\sum_{x'\in \D} \Normal{B y; B A x', \sigma^2 B B^{\top}}}$.
    \end{proposition}
\end{graybox}

The estimator $\xmmse$ is a weighted average of the training examples, where the weights $w(x \vert y)$ are proportional to the likelihood of observing $x$ given $y$.
\begin{remark}
    When $B \!=\! \Id$, $\xmmse$ is exactly the \emph{posterior mean}: $\xmmse(\vy) = \Mean{\vx \vert \vy}$.
\end{remark}
\paragraph{Physics-aware estimator} 
To elucidate the role of $B$, we remark that the weights have the following explicit form:
\begin{equation*}%\label{eq:empirical_mmse_with_B} 
    w(x \vert y) \propto \exp \left( -\frac{1}{2\sigma^2}\norm{\Pi_{\Im{B^\top}}(A(\bar{x} - x) + e)}^2 \right)
\end{equation*}
By selecting $B=A^+$ or $A^\top$, we ensure that $\Im{B^T} = \Im{A}$, then inner term is $A(\bar{x} - x) + \Pi_{\Im{A}}e$. 
The noise $e$ is projected onto the image space of $A$, reducing its magnitude. This projection enhances the signal-to-noise ratio, justifying the fact that physics-aware estimators often outperform physics-agnostic ones.
However, beyond the projection effect, the choice of $B$ manifests no other roles: the resulting MMSE estimator remains identical as long as B satisfies the condition $\Im{B^T} = \Im{A}$. We discuss this further in~\cref{subsec:agnostic_vs_aware}.

\paragraph{Memorization of the empirical MMSE} 
The weighrs $w(x \vert y)$ is non-negative and sums to $1$ over $x \in \D$, hence $\xmmse(y)$ is a convex combination of the training images, \i.e $\xmmse(y) \in \mathrm{conv}(\D)$ -- the convex envelope of the dataset $\D$.
In particular, as $\sigma \to 0$, $\xmmse(y)$ converges to its nearest neighbor  $x in \D$, in the sense that $BAx$ is closest to $By$, tie being averaged.
The MMSE $\xmmse$ therefore \emph{memorizes} the dataset $\D$: it reconstructs by re-weighting the training examples and never extrapolates outside $\mathrm{conv}(\D)$.
\begin{figure}[tb]
    \centering
    \def\size{0.16\linewidth}
    % ---------- DENOISING ----------------
    \rotatebox{90}{\hspace*{-0.5cm}\tiny denoising}
    \figwithlabel{$\bar{x}$}{\size}{./images/memorization/denoising_x.png}%
    \figwithlabel{$y$}{\size}{./images/memorization/denoising_y.png}%
    \figwithlabel{$\xmmse(y)$}{\size}{./images/memorization/denoising_x_mmse.png}%
    \figwithlabel{$\xtrans(y)$}{\size}{./images/memorization/denoising_x_equiv.png}%
    \figwithlabel{$\xtransloc(y)$}{\size}{./images/memorization/denoising_x_loc_equiv.png}%
    \figwithlabel{\tiny nearest neighbor}{\size}{./images/memorization/denoising_closest_train.png}%
    % ------------------ CONVOLUTION --------------------
    \\\rotatebox{90}{\hspace*{-0.6cm} \tiny deconvolution}
    \figwithsize{\size}{./images/memorization/conv_gaussian_0.5_x.png}%
    \figwithsize{\size}{./images/memorization/conv_gaussian_0.5_y.png}%
    \figwithsize{\size}{./images/memorization/conv_gaussian_0.5_x_mmse.png}%
    \figwithsize{\size}{./images/memorization/conv_gaussian_0.5_x_equiv.png}%
    \figwithsize{\size}{./images/memorization/conv_gaussian_0.5_x_loc_equiv.png}%
    \figwithsize{\size}{./images/memorization/conv_gaussian_0.5_closest_train.png}%
    \vskip -0.05in
    \caption{
    The MMSE and E-MMSE estimators memorize (yield the nearest neighbor), while the LE-MMSE estimator recombines training patches to give good reconstruction.    
    % Top: denoising. Bottom: deconvolution. 
    % The noise level is set to $\sigma=0.02$ and $\D$ contains 50k images from FFHQ at $64 \times 64$.
    }
    \label{fig:memorization_illustration}
    \vskip -0.15in
\end{figure}
\subsection{Constrained MMSE estimators}\label{sec:analytical_formula}
We first provide details on the constraint classes and then describe our main theoretical results: analytical formulas and properties of constrained MMSE estimators.
\subsubsection{Functional constraints: equivariance and locality}\label{subsec:constraint_classes}
We focus on CNN inverse problem solvers by considering function classes $\M$ that encode two core inductive biases -- \emph{equivariance} to geometric transformations~\citep{cohen2016group,veeling2018rotation} and \emph{locality} via finite receptive field~\citep{kadkhodaie2023learning}. 

We start with formal definition of these concepts. 
Let $x \in \R^N$ represent an image defined on a discrete 2D grid of size $H \times W = N$.  
\begin{graybox}
\begin{definition}[Translation equivariant]\label{def:trans_equiv_fn}
Let $\T = \Z_{H} \times \Z_{W}$ be the group of 2D cyclic translations, where $\Z_H = \{0, 1, \ldots, H-1\}$ and $\Z_W = \{0, 1, \ldots, W-1\}$.
For each $g = (g_h, g_w) \in \T$, its action is represented by a permutation matrix $T_g \in \R^{N \times N}$, where $T_g x$ shifts the image $x \in \R^N$ by $g_h$ pixels vertically and $g_w$ pixels horizontally with periodic boundary conditions.
A measurable map $\phi: \X \to \X$ is said to be \emph{translation equivariant} if it satisfies $\phi(T_g x) = T_g \phi(x)$ for all $x \in \X$ and $g \in \T$.
We denote by $\Mtrans$ the set of all such maps.
\end{definition}
\end{graybox}
We now add locality constraint.
For each pixel $n$,
consider the patch extractor $\Pi_n : x \in \R^N \mapsto \Pi_n x = x[\omega_n] \in \R^P$ that extracts a square patch $x[\omega_n]$ of size $\sqrt{P} \times \sqrt{P}$ centered at pixel $n$, and $\omega_n$ is the set of pixel indices in the patch with circular boundary conditions.

\begin{graybox}
\begin{definition}[Local and translation equivariant]\label{def:transloc_equiv_fn}
    A measurable map $\phi: \X \to \X$ is \emph{local and translation equivariant} if there exists a measurable map $f: \R^P \to \R$ such that, for all $x \in \X$ and any pixel $n$, the output of $\phi$ at pixel $n$  denoted by $\phi(x)[n]$, depends only on the local patch $x[\omega_n]$, that is $\phi(x)[n] = f(\Pi_n x)$.
    We denote by $\Mtransloc$ the set of all such functions.
\end{definition}
\end{graybox}

The class $\Mtransloc$ captures standard CNNs with small kernels and weight sharing or patch-based models, \eg MLPs acting on patches~\citep{glimpse_local}.

\subsubsection{Preliminary remarks}
\paragraph{Constraints and projection} The constrained MMSE has a simple geometric interpretation.
\begin{proposition}\label{prop:projection_mmse}
    For a closed set $\M$, the $\M$-constrained MMSE estimator in~\cref{def:constrained_mmse} is the \textbf{orthogonal projection} in $L^2$ of the posterior mean $\Mean{\vx \vert \vy}$ onto the subspace of random vectors of the form $\mathcal{X} = \{ \phi(B\vy) : \phi \in \M \}$, that is $\hat{x}_{\M}(\vy) = \Pi_{\mathcal{X}} \left( \Mean{\vx \vert \vy} \right)$.
\end{proposition}
\paragraph{Architecture equivariance and data augmentation}
A standard approach to obtain equivariant estimators is by using data augmentation: the set $\D$ is replaced by $\T(\D) = \{T_g x : x \in \D,\, g \in \T\}$ at training time. 
This promotes \emph{reconstruction equivariance}~\citep{chen2021equivariant} of $\hat{x}_{\M}$:
\begin{equation}
    \label{eq:reconstruction_equivariance}
    \hat{x}_{\M}(A T_g \bar{x} + e) = T_g \hat{x}_{\M}(A \bar{x} + T_g^{-1} e).
\end{equation} 
The \emph{data-augmented} MMSE estimators, are defined below. 
\begin{graybox}
    \begin{definition}[Data-augmented MMSE estimator]\label{def:data_augmented_mmse}
        The data-augmented MMSE estimator $\hat{x}_{\M}^{\text{\tiny aug}}$ is defined as the MMSE estimator $\hat{x}_\M$ with respect to the empirical measure on $\T(\D)$. 
    \end{definition}
\end{graybox}
    Reconstruction equivariance is often a desired property, but it differs from the \emph{structural equivariance} of $\phi$~\citep{chen2020group,nordenforsdata}, as illustrated in~\cref{fig:illustration_translation}.


\subsubsection{Translation Equivariant MMSE estimator} 
We begin with the closed-form of the E-MMSE estimator $\xtrans$ and its properties. 
\begin{graybox}
    \begin{theorem}[Closed-form of E-MMSE]\label{thm:equivariant_mmse}
        The E-MMSE estimator writes, for any $y\in \R^M$
        \begin{equation}\label{eq:formula_equivariant_mmse}
            \xtrans(y) = \sum_{x \in \D, g \in \T} T_g x \cdot w_g(x \vert y),
        \end{equation}
        where $w_g(x \vert y) \propto \Normal{T_g^{-1} By; B A x, \sigma^2 BB^\top}$. 
        % and the normalization constant is such that $\sum_{x \in \D, g \in \T} w_g(x \vert y) = 1$.  
    \end{theorem}
    
\end{graybox}

\begin{remark}
    This result holds true for any group $\G$ of orthogonal transformations, not just translations.
\end{remark}
The normalization constant of the weights $w_g(x \vert y)$ ensures that they sum to $1$ over all $x \in \D$ and $g \in \T$, we ignore it here for brevity. 
The E-MMSE estimator $\xtrans$ is a weighted average of the augmented dataset $\T(\D)$. 
In particular, for denoising ($A=B=\Id$), they coincide. %, and it was already observed in~\citep{kamb2025an}.
However, depending on the forward operator $A$ and the pre-inverse $B$, it can differ significantly from $\xmmseaug$.
Some properties of the E-MMSE $\xtrans$ are presented in~\cref{cor:equivariant_mmse_properties} and illustrated in~\cref{fig:illustration_translation}.
\begin{corollary}\label{cor:equivariant_mmse_properties}
    % \begin{graybox}
    The E-MMSE estimator $\xtrans$ in~\cref{thm:equivariant_mmse} satisfies the following properties:
	\begin{itemize}[leftmargin=*,itemsep=.2em,topsep=0.pt]
        \item The E-MMSE estimator $\xtrans$ \emph{memorizes} the augmented dataset: reconstructed images live in $\mathrm{conv}(\T(\D))$.
        \item Data-augmentation and architecture equivariance are identical ($\xtrans = \xmmseaug$) if and only if $A$ and $B$ are circular convolution, with $B$ invertible. In this case, the E-MMSE is reconstruction equivariant (satisfies~(\ref{eq:reconstruction_equivariance})), physics-agnostic and physics-aware solvers are identical.
    \end{itemize}
    % \end{graybox}
\end{corollary}


\begin{figure}[ht]
    \centering 
    \vskip -0.1in
    \includegraphics[trim = 5mm 0 0 0, clip, width=\columnwidth]{./tex_figure/translation.pdf}
    \vskip -0.1in
    \caption{
        Architectural equivariance does not always guarantee reconstruction equivariance~(\ref{eq:reconstruction_equivariance}).
        Input $y$ in 2nd row is shifted. The E-MMSE estimator $\xtrans$ is reconstruction equivariant for deconvolution but not inpainting.
        The LE-MMSE estimator $\xtransloc$ is reconstruction equivariant for deconvolution and shows reduced sensitivity to shifts for inpainting. 
    }\label{fig:illustration_translation}
    \vskip -0.1in
\end{figure}

\subsubsection{Locality and Translation Equivariance}
We now analyze the effect of local receptive fields.
\begin{graybox}
\begin{theorem}[Closed-form of LE-MMSE]\label{theorem:local_trans_estimator}
    Suppose that the $N$ matrices $Q_{n} \eqdef \Pi_n B \in \R^{P \times M}$ have the \emph{same rank} $r>0$ for any $n$\footnote{This assumption can be relaxed: a general formula by rank stratification is provided in the complete paper.}.
    The LE-MMSE estimator $\xtransloc$ admits, for any $y \in \R^M$ the following expression, defined pixel-wise for each pixel $n'$:  
    \vskip -0.1in
    \begin{equation}\label{eq:loc_equiv_formula}
        \xtransloc(y)[n'] = \sum_{x \in \D} \sum_{n = 1}^{N} x[n] \cdot w_{n', n}(x \vert y),
    \end{equation}
    where $w_{n', n}(x \vert y) \propto \Normal{Q_{n'} y; Q_n A x, \sigma^2 Q_n Q_n^\top}$. 
    % and the normalization constant is such that $\sum_{x \in \D} \sum_{n = 1}^{N} w_{n', n}(x \vert y) = 1$. 
\end{theorem}
\end{graybox}

The value at pixel $n'$ of the LE-MMSE estimator is a weighted average of \emph{all} the pixels in the training images.
This recombination can produce images outside $\mathrm{conv}(\D)$ or $\mathrm{conv}(\T(\D))$ (contrary to the MMSE or E-MMSE); see~\cref{fig:memorization_illustration} and~\cref{sec:experiments}.
The weights $w_{n', n}(x \vert y)$ can be seen as the posterior probability of the pixel $x[n]$ given the patch $(B y)[\omega_{n'}] = Q_{n'} y$. 

% \todo{Edouard: il faudrait repréciser qui sont $x$, $\bar{x}$ et $e$}
Recall that $y = A \bar{x} + e$, with $\bar{x}$ the signal to recover and  
let $\Delta_{n',n}(\bar x, x) \!\eqdef\! (B A \bar x) [\omega_{n'}] \!-\! (B A x) [\omega_n]$, 
the weights can be rewritten as  $w_{n', n}(x \vert y) \!\propto\! \exp\left( - \frac{\eta^2}{2\sigma^2}\right)$ with:
\begin{equation}\label{eq:weights_signal_vs_noise}
    \eta \eqdef \norm{Q_n^+ \Delta_{n',n}(\bar x, x) + Q_n^+ Q_{n'} e}.
\end{equation}
Hence, the weights measure the similarity between \emph{local patches} of measured-then-reconstructed images with the metric $Q_n^+$. It is perturbed by the noise term $Q_n^+ Q_{n'} e$, which is responsible for the estimator's variance.
% \cref{theorem:local_trans_estimator} generalizes the result from~\cite{kamb2025an}, which was derived for denoising.  
Some properties of the LE-MMSE $\xtransloc$ are given in~\cref{cor:local_trans_estimator_properties}. 
% \begin{graybox}
    \begin{corollary}\label{cor:local_trans_estimator_properties}
        The LE-MMSE estimator $\xtransloc$ in~\cref{theorem:local_trans_estimator} satisfies the following properties:
		\begin{itemize}[leftmargin=*,itemsep=.2em,topsep=0.pt]
            \item The LE-MMSE estimator $\xtransloc$ is a \emph{patchwork} of training patches. As $\sigma\! \to \! 0$, each pixel is set to the central pixel of the best matching patch from $\D$.    
            % \item In general, architectural equivariance is different from data augmentation: $\xtransloc \neq \xtransloc^{\mathrm{\tiny aug}}$.
            \item It is not a posterior mean: there exists no prior distribution of $\vx$ such that $\xtransloc(y)\!=\!\Mean{\vx \vert \vy \!=\! y}$.
        \end{itemize}
    \end{corollary}
% \end{graybox}

\subsubsection{Physics-agnostic and physics-aware estimators}\label{subsec:agnostic_vs_aware}

The expectation of $\eta^2$ in~(\ref{eq:weights_signal_vs_noise}) is given by
\begin{equation*}
    \mathbb{E}_{\ve}\left[\eta^2\right] = \norm{Q_n^+ \Delta_{n',n}(\bar x, x)}^2 + \sigma^2 \cdot \Tr \left( \mathrm{Cov}_{n',n}\right)
\end{equation*}
with $\mathrm{Cov}_{n',n} = Q_n^+ Q_{n'} Q_{n'}^\top Q_n^{+\top}$.    
This expression reveals that the pre-inverse $B$ plays two distinct roles:

\begin{itemize}[leftmargin=*,itemsep=.2em,topsep=0.pt]
	\item \emph{Signal discrimination}: the term $Q_n^+ \Delta_{n',n}(\bar x, x)$ measures the amplification/reduction of difference between dissimilar/similar patches for the pre-inverse $B$. %amplifies differences between dissimilar patches while suppressing differences between similar patches. 
A good choice of $B$ should improve this discrimination.
	\item \ \emph{Noise robustness}: the second term $\Tr \left( \mathrm{Cov}_{n',n}\right)$ measures noise amplification/reduction for $B$. A good choice should reduce this amplification.
\end{itemize}

Ideally, one would like to choose $B$ to optimize both criteria, but unfortunately they can be conflicting: preserving the signal $BAx\approx x$ may amplify the noise, in relation to the classical bias-variance decomposition. Most importantly this effect is operator dependent, as we now discuss.

For inpainting, choosing a physics-aware estimator with $B=A^+$ is an effective way to reduce the noise sensitivity while keeping signal's information. Indeed, with this choice, the noise within the masked region is eliminated, while the signal is preserved in the unmasked region.
For deconvolution, however, the situation is different. Choosing $B=A^+$ can result in significant noise amplification (\ie $\Tr \left( \mathrm{Cov}_{n',n}\right)$ is large), which is not counter-balanced by a better signal's discrimination. In this setting, a regularized inverse, balancing both aspects should likely be preferred.
This effect is illustrated in~\cref{fig:aware_vs_agnostic}.
When designing reconstruction algorithms, this trade-off between data-consistency and noise amplification must be carefully considered, in relation to the inverse problem at hand. 

\begin{figure}[ht]
    \centering
    \includegraphics[trim=8mm 0 0 0, clip,width=\columnwidth]{./tex_figure/aware_vs_agnostic.pdf}
    \vskip -0.1in
    \caption{
    %The operator $B$ should be chosen carefully to balance noise amplification and signal's discrimination trade-off.
	Physics-aware ($B\!=\!A^+$, bottom) has lower variance for inpainting (left), while physics-agnostic ($B\!=\!\Id$, top) has lower variance for deconvolution (right).
    Mean and pixel-wise variance are computed w.r.t $50$ noise realizations.         
    }
    \label{fig:aware_vs_agnostic}
    \vskip -0.15in
\end{figure}


\subsection{Numerical Experiments}\label{sec:experiments}
While the MMSE and E-MMSE estimators are interesting from a theoretical perspective, the LE-MMSE estimator is more relevant to practical neural network architectures used in imaging inverse problems.
We validate our theoretical findings about the LE-MMSE on three representative inverse problems: denoising, inpainting, and deconvolution. 
%\todo{We focus on LE-MMSE because \dots}
%{\color{blue} We focus on the LE-MMSE because it corresponds to one of the most widely used architectures in imaging inverse problems: CNNs with small kernels and weight sharing.}
\subsubsection{Experimental setup}    
    We implement three different local and translation equivariant neural network architectures: UNet2D~\cite{ronneberger2015u}, ResNet~\cite{he2016deep} and PatchMLP (an MLP acting on patches).
    % ; see~\cref{sec:supp_numerical} for architecture and training details.

    Implementing the analytical formula in~\cref{theorem:local_trans_estimator} requires computing a huge amount of pairwise distances, which is computationally demanding. We therefore restrict our experiments to $32\times 32$ color images and datasets of $10^4$ images. To accumulate Gaussian weights robustly and stably across batches, we use a careful online log-sum-exp implementation of the weighted averages (numerator and denominator). %, see~\cref{code:supp_python_code_locequiv_mmse,code:supp_online_sum_exp}. 
    % \todo{Je ne suis pas sûr que ce soit une bonne idée. J'aime bien personnellement, mais je trouve le relecteur moyen bête et méchant maintenant :). Edouard: Idem, on pourrait juste garder la première phrase et dire qu'on détaille les astuces d'implémentation dans l'appendice.}
    % \todo{Hai: Je les ai mis là au cas où, on peut les supprimer aussi. Moi je trouve que ça montre qu'on a bien implémenté la formule :)}

    For inpainting we use a square mask of size $15 \times 15$ at the center. For deconvolution, we use an isotropic Gaussian kernel of standard deviation of $1.0$.
    The noise level $\sigma$ is varying uniformly between $0$ and $1.0$ during training.


\subsubsection{Analytical formula and neural networks}\label{sec:neural_vs_analytical}
\paragraph{Case-by-case neural outputs prediction}
    We verify that trained networks approximate the LE-MMSE estimator, by comparing their outputs to the formula in~\cref{theorem:local_trans_estimator}.
    The PSNR between the trained UNet2D, the formula and the ground truth are reported in~\cref{fig:neural_vs_analytical_unet2d_patch_5}. 
    The reconstruction quality (orange and blue curves) degrades as noise level increases, which is expected as noise makes the problem harder.
    However, the PSNR between neural networks and the analytical formula (green curves) remains high ($\gtrsim 25$ dB) across all noise levels, indicating a strong alignment between the two.
    Similar behavior is observed in~\cref{tab:neural_vs_analytical_psnr} for other architectures and datasets. 
    \begin{figure}[ht]
        \centering
        \def\base{./images/neural_vs_analytical/localequivunet2dcondmodel_patch_5/plots}
        \vskip -0.05in
        \includegraphics[width=\linewidth]{\base/FFHQ_images32x32_subset_10000_combined_psnr_grid.pdf}
        \vskip -0.1in
        \caption{The green curves reveals a strong agreement (PSNR $\gtrsim\!25$ dB) between the trained UNet2D and the analytical formula for all inverse problems. 
        Median and interquartile range (IQR) using $50$ images per $\sigma$, $P = 5\times 5$ and $B\!=\!\Id$.
        }
        \label{fig:neural_vs_analytical_unet2d_patch_5} 
        \vskip -0.1in
    \end{figure}%
    We provide extensive numerical results on various settings (architectures, datasets, tasks, physics-agnostic/aware, patch sizes) in the complete paper. 
    These results confirm that \emph{trained neural networks closely approximate the analytical LE-MMSE estimator}.
    It is remarkable that different architectures consistently yield a similar output, as shown in~\cref{fig:teaser}. 
        
    \begin{table*}[ht]
        \centering
        \caption{
        \cref{theorem:local_trans_estimator} is verified across architectures, tasks, and datasets.
        We show the median of the PSNR between neural networks and the analytical formula with $P = 5 \times 5$ and $B\!=\!\Id$.
        The PSNR on FashionMNIST is consistently higher ($2\sim 3$ dB) than on FFHQ and CIFAR10, which we attribute to the lower complexity of FashionMNIST images.
        A lower PSNR on the test set in low-noise regimes is explained by the low-density regions.
        }
        \label{tab:neural_vs_analytical_psnr}
        \vspace*{-0.1cm}
        \includegraphics[width=.85\textwidth]{./images/neural_vs_analytical/tab_neural_vs_analytical_psnr_patch_5_side-by-side.pdf}
        \vskip -0.1in
    \end{table*}

\paragraph{Low-density regions}\label{sec:neural_generalization}
    A noticeable train-test gap remains in~\cref{fig:neural_vs_analytical_unet2d_patch_5,tab:neural_vs_analytical_psnr}, most pronounced at low noise level. 
    We attribute this to the \emph{measurement distribution} $p(y)$ induced by the empirical distribution.
    It is a Gaussian mixture, centered at the training measurements $Ax$: $p(y) = \frac{1}{|\D|} \sum_{x\in \D} \Normal{y; Ax, \sigma^2 \Id_M}$. 
    Its density is high near the centers $A x, x \in \D$, or for large noise levels $\sigma$, creating overlap between Gaussians. It becomes negligible far from the centers in low noise regimes.
    \begin{figure}[ht!]
        \centering
        \vskip -0.1in
        % \includegraphics[width=\columnwidth]{./images/neural_generalization/localequivunet2dcondmodel_patch_5/plots/multiop_FFHQ_images32x32_top5000_sigma_0.1_subset_10000_neg_log_density_psnr_identity_inform.pdf}
        \includegraphics[width=\columnwidth]{./images/neural_generalization/localequivunet2dcondmodel_patch_5/plots/multiop_FFHQ_images32x32_top5000_sigma_0.05_subset_10000_neg_log_density_psnr_identity_inform.pdf}
        \vskip -0.1in
        \caption{
            Higher density yield better alignment.
            We select test images from FFHQ-32, compute $-\!\log p(y)$ and plot it against the PSNR between the outputs of the LE-MMSE formula and a trained UNet2D, $\sigma=0.05$ and $P = 5 \times 5$. 
        }
        \label{fig:neural_generalization_unet2d}
        \vskip -0.1in
    \end{figure}%

    Both LE-MMSE and the networks are optimized with respect to the measurement distribution $p(y)$.   
    The networks and the theoretical formula best align in high-density regions of $p(y)$.
    This phenomenon is illustrated in~\cref{fig:neural_generalization_unet2d}, where we show that the alignment degrades as $-\!\log p(y)$ increases (or equivalently as $p(y)$ decreases). 
    Note that here we select 5000 test images with equally spaced $-\!\log p(y)$ values to cover a wide range of densities. 
    For large $\sigma$, the effect is less pronounced as the Gaussians overlap more.

    \paragraph{Out-of-distribution: how do networks generalize?}
    Neural networks often generalize well to out-of-distribution (OOD) data, but understanding this behavior remains an open question.  
    Interestingly, our analytical formulas provide insights to explain this phenomenon. 
    We consider here images $\bar{x}$ that lie in another dataset $\D'$ disjoint from the training set $\D$.  
    \begin{figure}[ht!]
        \centering
        \includegraphics[trim=17mm 0 0 0,clip,width=\linewidth]{./tex_figure/fig_qualitative_ood_unet2d_patch_5_ffhq.pdf}
        \caption{Our theory can predict the neural network outputs for out-of-distribution data: both UNet2D and the LE-MMSE formula are trained (or computed) on $\D = $ FFHQ-32. 
        They are tested on $\D' = $ CIFAR10.}
        \label{fig:ood_unet2d_patch_5}
        \vskip -0.1in
    \end{figure}
    In~\cref{fig:ood_unet2d_patch_5}, we compare the output of UNet2D and the LE-MMSE formula.
    %  on CIFAR10 images, while both are trained/computed on FFHQ-32.
    For large $\sigma$, we observe very close outputs between the neural network and the analytical formula.
    For small $\sigma$, as the OOD images lie in low-density regions of $p(y)$, the alignment degrades as expected from the previous discussion.
    However, even in this regime, the PSNR between the two outputs remains reasonably high.  
    This suggests that the network approximates the LE-MMSE estimator even on OOD data: generalization of the neural networks can be understood through the lens of the LE-MMSE estimator -- it leverages local patches from the training set to reconstruct unseen images. 


\subsubsection{Smoothed LE-MMSE and spectral bias}\label{subsec:smoothed_formula}
Deep neural networks also exhibit a \emph{spectral bias}~\citep{xu2019frequency,rahaman2019spectral}, \ie a preference for low-frequency functions, which yields smoother reconstructions in practice. 
Describing this spectral bias with functional constraints is missing from our derivation.
We model this effect by considering a \emph{randomized smoothing}~\citep{duchi2012randomized} variant: 
for a smoothing parameter $\epsilon > 0$ and $\vz \sim \Normal{0, \Id_M}$, define 
$\xtransloc^{\mathrm{sm}}(y) \eqdef \Mean{\xtrans(y + \epsilon \cdot \vz)}$. 
This averages the estimates over random perturbations, acting as a nonlinear low-pass filter that attenuates high-frequency variability in the output. 
\begin{figure}[ht!]
    \centering
    \vskip -0.1in
    \includegraphics[width=\columnwidth]{images/comparison_smoothed_vs_original/FFHQ_images32x32_subset_10000_sigma_0.05_patch_5_comparison_box.pdf}
    \vskip -0.1in
    \caption{
        Smoothed LE-MMSE ($\epsilon=0.05$) better matches neural network and improves reconstruction quality.}
        \label{fig:smoothed_vs_original}
    \vskip -0.1in
\end{figure}
It also mirrors standard neural network training, where different noise realizations are seen across epochs.
In~\cref{fig:smoothed_vs_original}, we compare the smoothed estimator $\xtransloc^{\mathrm{sm}}$ with the original $\xtransloc$, both computed with $P = 5 \times 5, \sigma=0.05$ and $B = \Id$ on FFHQ-32 images. 
We observe that the smoothed LE-MMSE better matches the neural network output ($1\sim3$ dB) and improves reconstruction quality ($\sim 1$dB).
This suggests that we can better capture neural network behavior by incorporating carefully designed smoothness constraints.

\subsection{Conclusion}
Through the lens of MMSE estimator in the context of imaging inverse problems, our work provides a theoretical framework to analyze the CNNs.
The strongly quantitative and qualitative agreement between our theoretical formulas and trained neural networks validates our LE-MMSE estimator. 
While being tractable and interpretable, our derived closed-form LE-MMSE offers valuable insights into the behavior of these networks under various settings: (1)the role of the foward operator $A$ and the pre-inverse $B$ on the resconstruction, (2)memorization does not contrardict generalization, i.e how these networks can extrapolate to out-of-distribution data by memorizing the local patches from the training set.
Our analysis also leverages the strength of neural networks in general by highlighting their ability to compress large datasets into their considerably smaller number of parameters, then rearticulate efficiently these parameters to approximate closely the heavily computational LE-MMSE estimator enabling fast and low-computational inference, explaining their practical success.
These findings open new avenues for further research, such as incorporating additional constraints to model other neural network such as smothness constrainsts to capture spectral bias, or permutation equivariance to model transformers~\citep{xu2024permutation}. The Gaussian assumption can be relaxed (e.g., exponential-family) by using the appropriate likelihood $p_{\vy \vert \vx}(y \vert x)$. 
Finally, beyond MSE, practical training often uses $\ell_1$ (yielding posterior median), perceptual, or mixed losses.
Analyzing the MMSE analogs under these alternatives is an interesting future venue.




\newpage
\section{Cyanobacteria recognition using deep learning techniques}\label{cyanoBacIdentification}
Cyanobacteria are essential components of aquatic ecosystems~\citep{singh2016cyanobacteria,diez2014ecological,saleem2025cyanobacteria}, driving global biogeochemical cycles through processes such as nitrogen fixation and oxygen production and carbon circulation. 
Their monitoring is critical for monitoring water quality, detecting harmful algal blooms, and assessing ecosystem dynamics.
However, the primary bottleneck in this process remains the rapid and accurate taxonomic identification of these organisms.
Traditional identification methods rely heavily on manual microscopic examination by experts, which is time-consuming, labor-intensive, subjective, and often impractical for large-scale monitoring efforts.
Various deep learning techniques have been proposed to automate the identification process, some focus with microscope image with a limited number of cyanobacteria species or genera \citep{blanco2025multimodal,baek2020identification}. Others focus on satellite images~\citep{kutser2004quantitative}, synthetic images~\citep{barrientos2023semantic} to detect cyanobacteria blooms.
To the best of our knowledge, there is no existing web-based application capable of identifying a wide taxonomic range of cyanobacteria from microscope images using deep learning.
To address this gap, we introduce CyDect (Cyanobacteria Detection), a web-based platform capable of detecting and classifying over \todo{number of classes} distinct cyanobacteria genera from microscope images. 
CyDect utilizes a Faster Region-based Convolutional Neural Network (Faster R-CNN) to handle the significant morphological variability of these organisms. 
The platform features a dual-interface architecture designed to bridge the gap between expert research and citizen science: it provides advanced analysis tools for biologists while simultaneously offering an accessible portal for the general public to upload images and visualize detection results.
\subsection{Data acquisition}
The dataset consists of microscope images of water samples containing cyanobacterias, these images are captured using high-resolution microscopes by our collaborators, biologists specialized in cyanobacteria study. To enhance the model robustness and generalization ability, the dataset includes images taken under various conditions, such as different lighting, magnification levels, and sample preparations. Each image is annotated with bounding boxes around individual cyanobacteria along with their corresponding species labels by our biologist experts. Initially, the dataset contains over \todo{number of images} images covering more than \todo{number of classes} different cyanobacteria species, but the dataset is continously enriched with new images and annotations as more samples are collected and processed.

\todo{Add more detail about the dataset}.

\subsection{Deep learning approach}
\subsubsection{Overview of RCNN family}
Faster-RCNN is a variant of the RCNN, which are a family of machine learning model for computer vision, and specifically object detection and localization. A RCNN model takes an image as input and then output bouding boxes around the detected objects along with their predicted class labels. In particular, a RCNN model taskes an input image, extracts region proposals that are likely to contain objects using Seletive Search algorithm \citep{uijlings2013selective}, computes features for each proposal using a learge convolutional neural network (CNN), then and then classifies each region using class-specific linear SVMSs \citep{girshick2014richfeaturehierarchiesaccurate}, as shown in Figure \ref{fig:rcnn_demo}. Originally, RCNN encounter several inefficiencies, one of these is training RCNN models involves multiple stages, including pre-training the CNN on a large dataset, fine-tuning it on the target dataset, and training class-specific SVMs, which makes the training process complex and time-consuming. Moreover, RCNN requires significant storage space as it needs to store the features for each region proposal, the model also suffers from botllenect as each region proposal need to be processed independently by the CNN, selective search for region proposals is also slow and not computationally expensive. These limitations are addressed by the fast-RCNN and faster-RCNN models which are presented below.
\begin{figure}[ht!]
    \centering
    \includegraphics[width=0.7\textwidth]{figures/RCNN_demonstration.png}
    \caption{RCNN demonstration. From \textit{Convolutional Neural Networks for Computer Vision} [Lecture notes] by Juliette Chevallier, 2025, INSA Toulouse.}
    \label{fig:rcnn_demo}
\end{figure}

Fast-RCNN response to this inefficiency by (1) performing feature extraction over the image before proposing regions, and (2) replaceing the SVM with a softmax layer \citep{girshick2015fastrcnn}. The first improvement is made by passing the entire image through an large CNN to generate feature maps, then region proposals are projected onto these feature maps as shown in Figure \ref{fig:fastrcnn_demo}. With these two improvements, fast-RCNN avoid the bottleneck as feature extraction is shared across region proposals, and a fast-RCNN is a unified model as the classifier is now part of the neural network, leading to a significant reduction in training and inference time.

\begin{figure}[ht!]
    \centering
    \includegraphics[width=0.7\textwidth]{figures/FastRCNN_demonstration.png}
    \caption{Fast-RCNN demonstration. From \textit{Convolutional Neural Networks for Computer Vision} [Lecture notes] by Juliette Chevallier, 2025, INSA Toulouse.}
    \label{fig:fastrcnn_demo}
\end{figure}


However, Fast-RCNN still rely on selective search for region proposals, which is not computationally efficient. Faster-RCNN solve this issue by replacing slow selective search with a region proposal network (RPN), which is a fully convolutional network that shares the convolutional layers with the object detection network \citep{ren2016fasterrcnnrealtimeobject}. The RPN takes the feature maps generated by the shared convolutional layers as input and outputs a set bounding box coordinates, each with an objectness score indicating the likelihood of containing an object, if this score is above a certain threshold, i.e the corresponding box has a good objectness, its coordinate get passed forward as a region proposal. The RPN uses a sliding window approach to generate boxes at different scales with certain fixed ratios, called anchor boxes, as shown in Figure \ref{fig:anchor_box}. By sharing the convolutional layers between the RPN and the object detection network, faster-RCNN significantly reduces the computational cost of region proposal generation, leading to faster inference times. The general architecture of a faster-RCNN is shown in Figure \ref{fig:fasterrcnn_demo}. 

\subsubsection{Faster-RCNN for cyanobacteria identification}
At the day this paper is written, the faster-RCNN model is no longer the state-of-the-art for object detection tasks, however, we still choose this model for this cyanobacteria identification task due to its robustness and high accuracy for small object detection, which is crucial as cyanobacterias are often small and densely packed in microscope images. Moreover, training faster-RCNN is easier compared to more recent models such as YOLO-family models and transformer-based models, which often require more complex training procedures, hyperparameter tuning and larger datasets to achieve good performance. To enhance the performance of our faster-RCNN model, we employ the ResNet (Residual Network) architecture integrating with Feature Pyramid Network (FPN) technique as the backbone network for feature extraction. This combination allows us to improve the model's ability to detect cyanobacterias at different scales, as we show now.



\begin{figure}[ht!]
    \centering
    % First Subfigure
    \begin{subfigure}[b]{0.4\textwidth}
        \centering
        \includegraphics[width=\textwidth]{figures/anchor_boxes.png}
        \caption{Anchor boxes used in the region proposal network \citep{ren2016fasterrcnnrealtimeobject}.}
        \label{fig:anchor_box}
    \end{subfigure}
    \hfill % Adds flexible space between images
    % Second Subfigure
    \begin{subfigure}[b]{0.55\textwidth}
        \centering
        \includegraphics[width=\textwidth]{figures/fasterrcnn_demonstration.png}
        \caption{Faster-RCNN demonstration. From \textit{Convolutional Neural Networks for Computer Vision} [Lecture notes] by Juliette Chevallier, 2025, INSA Toulouse.}
        \label{fig:fasterrcnn_demo}
    \end{subfigure}
    \caption{Anchor boxes and Faster-RCNN demonstration.}
    \label{fig:both_images}
\end{figure}

\paragraph{ResNet architecture}
ResNet (Residual Network) is a deep convolutional neural network architecture that introduces the concept of residual learning by stacking residual blocks.
The residual blocks learn residual functions with reference to the layer inputs, instead of learning unreferenced functions by connecting the input of a layer directly to its output via shortcut connections.
Formally, let $\mathbf{x}$ be the input to a certain layer, and let $\mathcal{F}(\mathbf{x})$ be the desired underlying mapping to be learned by the layer.
Instead of directly approximating $\mathcal{F}(\mathbf{x})$, residual blocks reformulate the mapping as $\mathcal{F}(\mathbf{x}) = \mathcal{H}(\mathbf{x}) - \mathbf{x}$, where $\mathcal{H}(\mathbf{x})$ is the original mapping.
The layer then learns the residual function $\mathcal{F}(\mathbf{x})$, and the output of the layer becomes $\mathcal{H}(\mathbf{x}) = \mathcal{F}(\mathbf{x}) + \mathbf{x}$, as illustrated in Figure \ref{fig:residual_block}.
This reformulation allows the network to learn identity mappings more easily, which helps to mitigate the vanishing gradient problem and enables the training of much deeper networks \citep{he2015deepresiduallearningimage}, often leading to improved performance on various computer vision tasks. 

\begin{figure}[ht!]
    \centering
    \includegraphics[width=0.5\textwidth]{figures/residual_block.png}
    \caption{An example of residual block in ResNet architecture \citep{he2015deepresiduallearningimage}.}
    \label{fig:residual_block}
\end{figure}

\paragraph{Feature Pyramid Network (FPN)}
Feature Pyramid Network (FPN) is a multi-scale feature extraction technique that enhances the ability of convolutional neural networks to detect objects at different scales. FPN constructs a feature pyramid by combining low-resolution, semantically strong features with high-resolution, semantically weak features through a top-down pathway and lateral connections \citep{lin2017featurepyramidnetworksobject}, as illustrated in Figure \ref{fig:fpn_architecture}. This approach allows the network to leverage both high-level semantic information and fine-grained spatial details, which is particularly beneficial for detecting objects with different scales like cyanobacterias in microscope images with varying magnifications.

\begin{figure}[ht!]
    \centering
    \includegraphics[width=0.8\textwidth]{figures/fpn_architecture.png}
    \caption{Feature Pyramid Network \citep{lin2017featurepyramidnetworksobject}: blue outlines denote feature maps, with thicker lines indicating stronger semantic features. A top-down pathway upsamples high-level features and combines them with corresponding lower-level maps through lateral connections, generating rich multi-scale representations.}
    \label{fig:fpn_architecture}
\end{figure}

\paragraph{Combining ResNet and FPN to form the backbone of Faster-RCNN}
We integrate the ResNet architecture with FPN to form the backbone of our faster-RCNN model.
The ResNet layers extract hierarchical features from the input images, while the FPN enhances these features by creating a multi-scale feature pyramid. 
This combination allows the faster-RCNN model to effectively detect cyanobacterias of various sizes and scales in microscope images, improving overall detection accuracy and robustness.
The logical Resnet-FPN backbone architecture is described as following: the input image is first feeded into the Resnet to extract feature maps at different levels, then these feature maps are passed through the FPN to generate a set of multi-scale feature maps, as shown in Figure~\ref{fig:fpn_architecture}. 
The anchor boxes in the RPN are then applied to these multi-scale feature maps in such a way that each level of the feature pyramid is responsible for detecting objects of specific scales. In particular, anchor boxes of a single scale are assigned to each level of the feature pyramid, since each level corresponds to a different spatial resolution, allowing the model to effectively capture objects of varying sizes. The overall architecture of the ResNet-FPN backbone within the faster-RCNN framework is illustrated in Figure~\ref{fig:resnet_fpn_combining}. 
\begin{figure}[ht!]
    \centering
    \includegraphics[width=0.8\textwidth]{figures/resnet_fpn.png}
    \caption{ResNet-FPN backbone within the Faster-RCNN framework: the input image is first processed by the ResNet to extract feature maps at different levels, then these feature maps are passed through the FPN to generate a set of multi-scale feature maps, which are then used by the RPN and the detection head for object detection.}
    \label{fig:resnet_fpn_combining}
\end{figure}

\subsection{Application deployment}
Our application is desgined to serve two distinct user groups: experts (biologists, researchers, taxonomists) and the general public. It consists of three main components: the expert interface, the general public interface, and the model backend API, all these components are hosted on a cloud server to ensure accessibility and scalability.
These elements are described in detail below.


\paragraph{API backend server}
The API backend server is built using \href{https://flask.palletsprojects.com/en/stable/}{Flask}. The server hosts the trained faster-RCNN model and the model training scripts. 
When users request predictions, the corresponding images are sent to the API backend server via an HTTP POST request, The server processes the image using the faster-RCNN model, generates detection results, and sends them back to the user interface in JSON format for display.

\paragraph{General user interface} The general user interface is developed using \href{https://dash.plotly.com/}{Plotly Dash}, a popular framework for building interactive web applications in Python.
The interface is desgined to be free-access, user-friendly and does not require users to have any technical background.
It allows users to upload microscope images and view the detection results generated by the faster-RCNN model. Moreover, users can provide feedback on the detection results by correcting misidentified cyanobacteria or adding missing annotations. This feedback is collected and stored in a database, waiting for expert validation to ensure the quality and accuracy of the annotations.\todo{Add image of this interface}


\paragraph{Expert interface} We use \href{https://labelstud.io/}{Label Studio}, an open-source data labeling tool, for the expert interface. The access to this interface is restricted to authorized experts only.
Experts can upload, annotate new microscope images, validate user corrections from the general public interface, and manage the dataset effectively.
The annnotation process is facilitated by enabling pre-annotations using the current faster-RCNN model, allowing experts to review and adjust the model's predictions rather than starting from scratch.
Additionally, they can also trigger the model retraining process, ensuring that the backend model remains up-to-date with the latest annotated data and continuously improves its performance.\todo{Add image of this interface} 

\paragraph{Hosting server} The application is hosted on a cloud server provided by \href{https://drocc.fr/crocc/}{Cloud Recherche Occitanie (CROCC)}.

In general, the application workflow is as follows: users upload microscope images through either the expert interface or the general public interface, these images are then sent to the API backend server where the faster-RCNN model process them and return the detection results to the user interfaces for display; experts can also manage and enrich the dataset by uploading their own microscope images or/and validating user-correction; the model retraining proccess can only be triggered by experts, as shown in Figure~\ref{fig:app_workflow}.
\begin{figure}[ht!]
    \centering
    \includegraphics[width=0.8\textwidth]{figures/app_workflow.png}
    \caption{Application workflow diagram: users upload microscope images through either the expert interface or the general public interface, these images are then sent to the API backend server where the faster-RCNN model process them and return the detection results to the user interfaces for display; experts can also manage and enrich the dataset by uploading their own microscope images or/and validating user-correction; the model retraining proccess can only be triggered by experts.}
    \label{fig:app_workflow}
\end{figure}
\subsection{Preliminary results}
\subsection{Conclusion and future work}


\newpage
\bibliography{biblio.bib}


\newpage
\appendix
\onecolumn
{\LARGE{Annexes for the first project -- An analytical theory ofconvolutional neural networks in imaging inverse problems}}
\section{Mathematical formalism}
\subsection{Preliminaries}\label{sec:supp_preliminaries}
\paragraph{Notation conventions}
In what follows, we use the following notation:
\begin{itemize}
    \item Sets are indicated by calligraphic letters, \eg $\D, \M, \T, \G$.
    \item Random vectors are denoted by bold lowercase letters, \eg $\vx, \vy, \vz$.
    \item $A \in \R^{M \times N}$ denotes a linear operator from $\R^N$ to $\R^M$ (\eg convolution, inpainting).
    \item $\Normal{x; \mu, \Sigma}$ denotes the probability density function (PDF) of a Gaussian distribution with mean $\mu$ and covariance $\Sigma$ evaluated at $x$. The covariance matrix $\Sigma$ can be non-singular or singular (see \cref{def:supp_degenerate_gaussian}). 
    \item The $n$-th coordinate of a vector $x\in \R^N$ is denoted either $x_n$ or $x[n]$. Similarly, the coordinates of a multivalued function $\phi:\R^N\to \R^N$  can be denoted either $\phi_n(x)$ or $\phi(x)[n]$.
    \item The empirical data distribution $\ptrain$ is defined by 
    \begin{equation}
        \ptrain = \frac{1}{|\D|} \sum_{x\in \D} \delta_{x},   
    \end{equation} 
    where $\D$ is the training dataset of finite size $|\D|$.
\end{itemize}

\paragraph{Degenerate Gaussian distribution}
We can define the multivariate Gaussian distribution for positive semi-definite covariance
matrices~\citep{rao1973linear}. Below, we recall a simple definition of the positive semi-definite covariance 
multivariate Gaussian, sometimes called the degenerate or singular multivariate Gaussian.
\begin{definition}[Degenerate Gaussian distribution]\label{def:supp_degenerate_gaussian}
    A random vector $\vz \in \R^N$ has a Gaussian distribution with mean $\mu \in \R^N$ and covariance matrix $\Sigma \in \R^{N \times N}, \Sigma \succeq 0$ if its probability density function (PDF) is given by:
    \begin{equation}
        \Normal{z; \mu, \Sigma} = \frac{1}{(2\pi)^{r/2} \sqrt{|\Sigma|_{+}}} \exp\left(-\frac{1}{2} (z - \mu)^\top \Sigma^{+} (z - \mu)\right) \mathbbm{1}_{\supp{\vz}}(z),
    \end{equation}
    on the support $z \in \supp{\vz} = \mu + \Im{\Sigma}$ and zero elsewhere.  
    In this equation, $r = \mathrm{rank}(\Sigma)$, $\Sigma^{+}$ denotes the Moore-Penrose pseudo-inverse of $\Sigma$ and $|\Sigma|_{+}$ is the pseudo-determinant of $\Sigma$ (product of non-zero eigenvalues). 
\end{definition}
The support $\supp{\vz}$ of the distribution is the $r-$ dimensional affine subspace of $\R^N$, $r$ is the rank of $\Sigma$. The density is with respect to the Lebesgue measure restricted to $\supp{\vz}$, constructed via the pushforward measure of the standard Lebesgue measure on $\R^r$ by the affine map $v \mapsto \mu + \Sigma_r v$, where $\Sigma = \Sigma_r \Sigma_r^\top$.  
This measure also coincides (up to a normalization constant, equal to the pseudo-determinant of $\Sigma$) with the $r-$dimensional Hausdorff measure. 
For completeness, we provide a derivation of the density of $\vz$ with respect to the $r$-dimensional Hausdorff measure $\mathcal{H}^r$ on the affine support $\supp{\vz}$.

    Suppose $\Sigma_r \in \mathbb{R}^{n \times r}$ satisfies $ \Sigma = \Sigma_r \Sigma_r^\top $ and $\mathrm{rank}(\Sigma) = r$.
    Define the random vector $\vz = \mu + \Sigma_r \vw$, where $\vw \sim \mathcal{N}(0, I_r)$.
    Consider the affine map $ \Phi: \mathbb{R}^r \to \mathbb{R}^N $ defined by $ \Phi(w) = \mu + \Sigma_r w $.
    This map is a smooth bijection from $\mathbb{R}^r$ onto the support $\supp{\vz} = \mu + \Im{\Sigma}$.

    The probability measure of $\vz$, denoted $\mathbb{P}_{\vz}$, is the pushforward of the standard Gaussian measure $\gamma^r$ via $\Phi$. The standard Gaussian density on $\mathbb{R}^r$ is $g(w) = (2\pi)^{-r/2} e^{-\frac{1}{2} \|w\|^2}$.
    For any measurable set $A \subset \supp{\vz}$, we have:
    \begin{equation}
        \mathbb{P}_{\vz}(A) = \gamma^r(\Phi^{-1}(A)) = \int_{\Phi^{-1}(A)} g(w) \, d\lambda^r(w).
    \end{equation}
    To find the density with respect to the Hausdorff measure $\mathcal{H}^r$ (the standard volume measure on the support), we apply the Area Formula~\citep{federer1996geometric} (change of variables). The Jacobian determinant of $\Phi$ is:
    $$ J\Phi = \sqrt{\det(\Sigma_r^\top \Sigma_r)} = \sqrt{|\Sigma|_+}. $$
    Using the change of variables $z = \Phi(w)$, the volume elements relate via $d\mathcal{H}^r(z) = J\Phi \, d\lambda^r(w)$. Therefore:
    $$ d\lambda^r(w) = \frac{1}{\sqrt{|\Sigma|_+}} d\mathcal{H}^r(z). $$
    Substituting this into the integral and using $w = \Phi^{-1}(z) = \Sigma_r^+ (z - \mu)$:
    \begin{align*}
        \mathbb{P}_{\vz}(A) &= \int_{A} g(\Phi^{-1}(z)) \frac{1}{\sqrt{|\Sigma|_+}} d\mathcal{H}^r(z) \\
        &= \int_{A} \frac{1}{(2\pi)^{r/2}\sqrt{|\Sigma|_+}} \exp\left( -\frac{1}{2} \|\Sigma_r^+ (z - \mu)\|^2 \right) d\mathcal{H}^r(z).
    \end{align*}
    Noting that $\|\Sigma_r^+ (z - \mu)\|^2 = (z - \mu)^\top (\Sigma_r^+)^\top \Sigma_r^+ (z - \mu) = (z - \mu)^\top \Sigma^+ (z - \mu)$, we recover the density given in Definition \ref{def:supp_degenerate_gaussian}.

When the covariance matrix is non-singular, we recover the standard PDF of a Gaussian distribution. 

\paragraph{Degenerate Multivariate Gaussian and Mahalanobis Distance}
The Mahalanobis distance quantifies the distance from $z$ to $\mu$ relative to the covariance structure and is defined as:

\begin{equation}
D_\Sigma(z, \mu) = \sqrt{(z - \mu)^\top \Sigma^+ (z - \mu)}.
\end{equation}

When $z \not\in \supp{\vz}$, the density is zero, corresponding to an infinite effective distance outside the support.

\begin{itemize}
    \item Eigenvalue Decomposition. Consider the spectral decomposition of the covariance matrix:
    \begin{equation*}
        \Sigma = U \Lambda U^\top,
    \end{equation*}
    where:
    \begin{itemize}
        \item $U$ is an orthogonal matrix whose columns are the eigenvectors of $\Sigma$.
        \item $\Lambda = \diag(\lambda_1, \lambda_2, \dots, \lambda_d)$ is a diagonal matrix with eigenvalues $\lambda_1 \geq \lambda_2 \geq \dots \geq \lambda_r > 0 = \lambda_{r+1} = \dots = \lambda_d$, sorted in descending order, with zeros corresponding to the degenerate directions.
    \end{itemize}
    The pseudo-inverse is:
    \begin{equation}
        \Sigma^+ = U \Lambda^+ U^\top,
    \end{equation}
    where $\Lambda^+ = \diag(1/\lambda_1, \dots, 1/\lambda_r, 0, \dots, 0)$.

    Transform the coordinates to the eigen-basis by defining $y = U^\top (x - \mu)$. The Mahalanobis distance simplifies to:

    \begin{equation}
        D_\Sigma(x, \mu) = \sqrt{y^\top \Lambda^+ y} = \sqrt{\sum_{i=1}^r \frac{y_i^2}{\lambda_i}},
    \end{equation}
    since $y_i = 0$ for $i > r$ on the support $\supp{\vz}$. The exponent in the density becomes:   
    \begin{equation}
        -\frac{1}{2} \sum_{i=1}^r \frac{y_i^2}{\lambda_i}.
    \end{equation}

    \item Distance in Different Directions

    The effect of deviations from the mean $\mu $ along the directions of the eigenvectors (principal axes) depends on the corresponding eigenvalues:

    \begin{itemize}
        \item \textbf{Directions with large eigenvalues ($\lambda_i \gg 0 $)}: The term $\frac{y_i^2}{\lambda_i} $ grows slowly as $|y_i| $ increases. Deviations along these high-variance directions contribute less to reducing the density, effectively scaling the distance by $\sqrt{\lambda_i} $. This allows larger Euclidean deviations in these directions.
        \item \textbf{Directions with small positive eigenvalues ($0 < \lambda_i \ll 1 $)}: The term $\frac{y_i^2}{\lambda_i} $ grows rapidly even for small $|y_i| $. Deviations are heavily penalized, scaling the distance by $1 / \sqrt{\lambda_i} $, making small deviations appear ``far'' in the Mahalanobis sense.
        \item \textbf{Directions with zero eigenvalues ($\lambda_i = 0 $)}: These correspond to the null space of $\Sigma$. Here, $y_i $ must be exactly zero for the density to be non-zero, enforced by the Dirac delta measure. Any non-zero deviation results in zero density, equivalent to an infinite Mahalanobis distance, indicating no variability in these directions.
    \end{itemize}
\end{itemize}

\paragraph{Integration on linear subspaces}
The following lemma is useful for change of variable with linear mapping \citep{federer1996geometric}.   
\begin{lemma}[Integration on linear subspaces]\label{lemma:supp_integration_linear_subspace}
    Let $\Normal{y; \mu, \Sigma}$ be the PDF of a Gaussian distribution in $\R^N$, with mean $\mu \in \R^N$ and covariance matrix $\Sigma \in \R^{N \times N}$ of rank $r_0 \leq N$. 
    For any linear application $B: \R^N \to \R^P$, and any measurable $f$, we have:
        \begin{equation}
            \int_{\R^N} f(B y) \Normal{y; \mu, \Sigma} d \Haus^{r_0}(y) = \int_{\R^P} f(v) \Normal{ v; B \mu, B \Sigma B^\top} d \Haus^{r}(v),   
        \end{equation}
        where $\Normal{ v; B \mu, B \Sigma B^\top}$ is the PDF of a (possibly degenerate) Gaussian distribution with respect to the $r-$dimensional Hausdorff measure $\Haus^{r}$, with $r$ being the rank of $B \Sigma B^\top$.
        This PDF is supported on the $r-$dimensional subspace $B \mu + \Im{B \Sigma B^\top} \subset \Im{B} \subset \R^P$. 
\end{lemma}
\begin{proof}
    The proof of this result is relatively straightforward by using the density of a degenerate Gaussian distribution \cref{def:supp_degenerate_gaussian}. 
    Let $\vy \sim \Normal{\mu, \Sigma}$ and $\vz = B \vy$, then $\vz \sim \Normal{B \mu, B \Sigma B^\top}$. We have
    \begin{align}
        \int_{\R^N} f(B y) \Normal{y; \mu, \Sigma} d \Haus^{r_0}(y) &= \Mean{f(B \vy)} \\&= \Mean{f(\vz)} \\&= \int_{\R^P} f(v) \Normal{ v; B \mu, B \Sigma B^\top} d \Haus^{r}(v).
    \end{align}
\end{proof}
\begin{remark}
    When $\Sigma$ is full rank, the Hausdorff measure $\Haus^{r_0}$ coincides with the Lebesgue measure on $\R^N$. 
\end{remark}

\subsection{Minimum Mean Square Error (MMSE) estimator}
The MMSE estimator is the optimal estimator in the following sense
\begin{definition}[MMSE estimator]\label{def:supp_mmse}
    Given two random vectors $\vx \in \X, \vy \in \Y$ and a linear operator $B: \Y \to \X$.
    The MMSE estimator of $\vx$ given $B \vy$ is the best approximation \emph{random variable} $\phi^\star(B \vy)$ to $\vx$, in the least-square sense: 
    \begin{equation}
        \xmmse = \phi^\star \circ B \quad \text{where} \quad \phi^{\star} = \argmin_{\phi : \X \to \X} \Mean{ \norm{\phi(B\vy) - \vx}^2}
    \end{equation}
\end{definition}
It is well-known that the MMSE estimator coincides with the \emph{conditional expectation}. That is, for any $y$ such that $p(y) > 0$, we have: 
\begin{equation}
    \xmmse(y) = \Mean{\vx \vert B \vy = B y} = \int x \cdot p(x \vert B y) dx 
\end{equation}
Composing $\xmmse$ with the random variable $\vy$, we get
$$\xmmse(\vy) = \Mean{\vx \vert B \vy}.$$
In particular, if $B\!=\!\Id$ (in this case $M = N$), the MMSE estimator reduces to the classical \emph{posterior mean}: $\xmmse(y) = \Mean{\vx \vert \vy = y}$. 
Note that both $\Mean{\vx \vert \vy = y}$ and $\Mean{\vx \vert \vy}$ are often called condition expectation, but these are different objects. 
In particular, $\Mean{\vx \vert \vy = \cdot} = \xmmse(\cdot)$ is a function $\Y \to \X$ while $\Mean{\vx \vert \vy}$ is a random variable assuming values in $\X$. 
However, finding the MMSE estimator amounts to finding the optimal function $\phi^\star$.

\subsection{Constrained Minimum Mean Square Error (MMSE) estimator}
The classical MMSE estimator is defined as the best approximation function over the space of measurable function from $\Y$ to $\X$. 
When adding functional constraints to the estimator, we would like to find the best approximation function over a subspace $\M$. 
\begin{equation}
    \min_{\phi \in \M} \Mean{ \norm{\phi(B \vy) - \vx}^2}
\end{equation}
For example, the subspace $\M$ could be the set of measurable functions from $\X \to \X$ and translation equivariant. 
\subsection{Optimality condition}
We first state a simple first-order sufficient and necessary optimality condition for solving the Constrained MMSE. 
\begin{proposition}[Optimality condition]\label{prop:supp_optimality_condition}
    Let $\vx \in \X, \vy \in \Y$ be two random variables and $\M$ be a vector space of measurable functions from $\X$ to $\X$, $B$ be a linear operator from $\Y$ to $\X$. 
    Then $\phi^\star$ is a minimizer of 
    \begin{equation*}
        \min_{\phi \in \M} \Mean{ \norm{\phi(B \vy) - \vx}^2}
    \end{equation*} 
    if and only if
    \begin{equation}
        \Mean{\inner{\varphi(B \vy), \phi^\star(B \vy) - \vx}} = 0 \qquad \text{ for all } \varphi \in \M.
    \end{equation}
    
\end{proposition}
\begin{proof}
    Let $J(\phi) = \Mean{ \norm{\phi(B \vy) - \vx}^2}$. For all $t \in \R$ and for all $\varphi \in \M$, we have
    \begin{align*}
        J(\phi^\star + t \varphi) &= \Mean{\norm{(\phi^\star + t \varphi) (B \vy) - \vx}^2} \\
        &= J(\phi^\star) + 2t \underbrace{\Mean{\inner{\varphi(B \vy), \phi^\star(B \vy) - \vx}} }_{a}  + t^2  \underbrace{ \Mean{\norm{\varphi(B \vy)}^2}}_{b}\\
        &= J(\phi^\star) + 2a t + b t^2 
    \end{align*}
    Therefore, $\phi^\star$ is a minimizer of $J$ on $\M$ if and only if $J(\phi^\star + t \varphi) \geq J(\phi^\star)$ for all $t \in \R$ and all $\varphi \in \M$.
    This is equivalent to the condition that $2a t + b t^2  \geq 0$, for all $t \in \R$ and all $\varphi \in \M$. 
    Since this difference term is a quadratic function in $t$ and $b \geq 0$, it is non-negative for all $t \in \R$ if and only if $a = 0$.
    Therefore, we have the sufficient and necessary condition that $\Mean{\inner{\varphi(B \vy), \phi^\star(B \vy) - \vx}} = 0$ for all $\varphi \in \M$.
\end{proof}

The optimality condition \cref{prop:supp_optimality_condition} states that the residual $\phi^\star(B \vy) - \vx$ is orthogonal, in the $L^2$ sense, to every perturbation in the feasible set $\M$. Equivalently, $\phi^\star(B \vy)$ is the orthogonal projection of $\vx$ onto $\M$ in the $L^2$ sense. When $\M$ is the space of all square-integrable functions of $\vy$ (\ie no constraints) and $B=\Id$, this projection yields the classical MMSE estimator (posterior mean), $\Mean{\vx \vert \vy}$. In the constrained case, $\phi^\star(B \vy)$ can also be viewed as the orthogonal projection of the conditional expectation $\Mean{\vx \vert \vy}$ onto the subspace $\{\phi(B \y) : \phi \in \M \}$.
\begin{proposition}\label{prop:supp_projection_mmse}[\cref{prop:projection_mmse} in the main paper]
    Given a closed set $\M$. 
    The $\M$-constrained MMSE estimator in~\cref{def:constrained_mmse} is the orthogonal projection (in $L^2$ sense) of the posterior mean $\Mean{\vx \vert \vy}$ onto the subspace of $\vy$-measurable random vectors of form $\mathcal{X} = \{ \phi(B\vy) : \phi \in \M \}$.
    That is,
    $$
        \hat{x}_{\M}(\vy) = \Pi_{\mathcal{X}} \, \Mean{\vx \vert \vy}.
    $$
\end{proposition}
\begin{proof}%[Proof of~\cref{prop:projection_mmse}]
The classes $\Mtrans$ and $\Mtransloc$ are linear subspaces of the space of measurable functions from $\R^N$ to $\R^N$: they are closed under addition and scalar multiplication, hence the projection is well-defined.
The MMSE estimator in~\cref{prop:expression_MMSE} is the posterior mean $\Mean{\x \vert \y}$, which is the orthogonal projection of $\x$ onto the space of $\y$-measurable random vectors. 
Similarly, the $\M-$constrained MMSE estimator is the projection of $\x$ on to the subspace $\{\phi(B \y) : \phi \in \M \}$. 
Using the Pythagorean decomposition, for any $\phi \in \M$, we have
\begin{equation}\label{eq:supp_pythagorean_decomposition}
    \Mean{\norm{\phi(B \y) - \x}^2} = \Mean{\norm{\phi(B \y) - \Mean{\x \vert \y}}^2} + \Mean{\norm{\Mean{\x \vert \y} - \x}^2}
\end{equation}  
Therefore
\begin{equation}
    \argmin_{\phi \in \M} \, \Mean{\norm{\phi(B \y) - \x}^2} = \argmin_{\phi \in \M} \, \Mean{\norm{\phi(B \y) - \Mean{\x \vert \y}}^2}
\end{equation}
and the $\M-$constrained MMSE in~\cref{def:constrained_mmse} is the projection of the posterior mean $\Mean{\x \vert \y}$ onto the subspace $\{\phi(B \y) : \phi \in \M \}$.    

Moreover, the decomposition in~\cref{eq:supp_pythagorean_decomposition} has interesting interpretation: it isolates the sources of error in the estimation. 
The first term $\Mean{\norm{\phi(B \y) - \Mean{\x \vert \y}}^2}$ is the approximation error due to the restriction to the subspace $\M$, while the second term $\Mean{\norm{\Mean{\x \vert \y} - \x}^2}$ is the irreducible Bayes error.
\end{proof}

\subsection{Structural constraint: equivariant functions}
\label{sub:definition_equiv}
The below definitions are taken from \citep{Celledoni_2021_equivariant_neural_networks}. 
\begin{definition}[Group] A \emph{group}, to be denoted $\G$, is a set equipped with an associative operator $\cdot : \G \times \G \to \G$, which satisfies the following conditions:
    \begin{enumerate}
        \item If $g_1, g_2 \in \G$ then $g_2 \cdot g_1 \in \G$
        \item If $g_1, g_2, g_3 \in \G$ then $(g_1 \cdot g_2) \cdot g_3 = g_1 \cdot (g_2 \cdot g_3)$
        \item There exists $\id \in \G$ such that $e \cdot g = g \cdot \id = g$ for all $g \in \G$.
        \item If $g \in \G$ there exists $g^{-1} \in \G$ such that $g^{-1} \cdot g = g \cdot g^{-1} = \id$.
    \end{enumerate}    
\end{definition}
\begin{definition}[Group action]
    Given a group $\G$ and a set $\X \subset \R^N$, we say that $\G$ acts on $\X$ if there exists a function $T: \G \times \X \to \X$ (we denote by $T_g(x)$ for $g \in \G$ and $x \in \X$) that satisfies:
    \begin{equation*}
        T_{g_1} \circ T_{g_2} = T_{g_1 \cdot g_2} \qquad \text{and} \qquad T_{\id} = \mathrm{id}
    \end{equation*} 
\end{definition}
Given a general group $\G$. A function $\phi: \X \to \X$ and group action $T$ of $\G$ on $\X$. A function $\phi$ is called $\G$-\emph{equivariant} if it satisfies
\begin{equation*}
    \phi(T_g(y)) = T_g \phi(y) \qquad \text{for all } x \in \X \text{ and for all } g \in \G.
\end{equation*}
\begin{proposition}\label{prop:supp_reynolds_averaging}
    If the group $\G$ is finite, the following properties hold true:
    \begin{enumerate}
        \item \textbf{Invariance}: for any function $\phi: \X \to \R$, the function $\bar{\phi} = \sum_{g} \phi \circ T_g$ is invariant. (And similarly for function defined in $\Y$).  
        \item \textbf{Equivariance}: for any function $\phi: \X \to \X$, the function $\bar{\phi} = \sum_{g} T_g^{-1} \circ \phi \circ T_g$ is equivariant. This is called \textbf{Reynolds averaging}.
    \end{enumerate} 
\end{proposition}

For image data $x \in \R^N$, let $H, W \in \mathbb{N}$ be the dimensions of a discrete grid $\Omega = \Z_H \times \Z_W$ with $H \times W = N$. 
\begin{definition}[Translation equivariant functions]
    Let $\T = \Z_H \times \Z_W$ be the group of 2D cyclic translations. 
    For every group element $g = (g_h, g_v) \in \T$, we define the \emph{translation operator} $T_g: \R^N \to \R^N$ as the permutation matrix that acts on an image $x \in \R^N$ by shifting its indices:
    \[
        (T_g x)[i, j] = x[(i - g_h) \bmod H, \, (j - g_v) \bmod W]
    \]
    for all $(i, j) \in \Omega$.
    A measurable map $\phi: \R^N \to \R^N$ is said to be \emph{translation equivariant} if it commutes with the translation operator for all $g \in \T$:
    \[
        \phi(T_g x) = T_g \phi(x) \qquad \text{for all } x \in \R^N.
    \]
\end{definition}


\subsection{Structural constraints: local and translation equivariant functions}
\label{sub:definition_local_equiv}
We first define precisely the patch extractor. 
Let $n \in \Omega$ be a pixel coordinate on the grid. We define the \emph{patch extractor} $\Pi_n: x \in \X \to \Pi_n x = x[\omega_n] \in \R^P$ which extracts a square patch $x[\omega_n]$ of size $\sqrt{P} \times \sqrt{P}$ centered at $n$.
The extraction uses circular boundary conditions, such that the patch is given by the grid values at indices:
\[
    \omega_n = \{ n + \delta \pmod{(H, W)} \mid \delta \in \Delta \}
\]
where $\Delta$ is the set of offsets defining the square neighborhood centered at zero.

\begin{definition}[Local and translation-equivariant functions]
\label{def:supp_transloc_equiv_fn}
    A measurable map $\phi: \X \to \X$ is said to be \emph{local} if it can be represented as a sliding window operation. 
    Specifically, if there exist a measurable function $f: \R^P \to \R$ such that for all $x \in \X$ and all $n \in \Omega$:
    \[
        \phi(x)[n] = f(\Pi_n x).
    \]
    By construction, any such function $\phi$ is also translation equivariant due to the circular boundary conditions of the patch extractor $\Pi_n$.
    We denote by $\Mtransloc$ the set of all such maps.
\end{definition}

This class captures standard CNNs with finite kernels and weight sharing or patch-based methods, \eg MLPs acting on patches~\citep{glimpse_local}: the output of $\phi$ at pixel $n$, denoted $\phi(x)[n]$ or $\phi_n(x)$, depends only on the local patch (receptive field) $x[\omega_n]$. 
Note that by construction, functions in \cref{def:supp_transloc_equiv_fn} are translation equivariant.


% ############################################################
% ################          THEORY         ###################
% ############################################################

\section{Analytical solution for inverse problems}
In the following, we will derive the analytical solution for the MMSE estimator under these constraints.

Consider a random variable $\vx \sim \pdata, \vx \in \X$ and the forward (measurement) model
\begin{equation}
    \vy = A \vx + \ve \qquad \text{ where } \qquad \ve \sim \Normal{0, \sigma^2 \Id}.
\end{equation}
We would like to find the MMSE estimator of $\vx$ given $\vy$, with or without constraints. 
In this section, we will derive the analytical solution for the MMSE estimator under various constraints, when the true underlying distribution of $\vx$ is replaced by the empirical distribution $\ptrain$. 
Under Gaussian noise, the likelihood of the measurement $\vy$ given $\vx$ is given by
\begin{equation}
    p(y \vert x) = \Normal{y; A x, \sigma^2 \Id} \propto \exp\left( -\frac{\norm{y - A x}^2}{2 \sigma^2} \right).
\end{equation}
Therefore, for any function $\phi^{\star}$ and $\varphi$, we have
\begin{align}
    \Mean{\inner{\varphi(B \vy), \phi^\star(B \vy) - \vx}} &= \E_{\x \sim \ptrain} \E_{\vy \vert \vx} \left[ \inner{\varphi(B \vy), \phi^\star(B \vy) - \vx} \right] \notag \\
    &= \frac{1}{|\D|} \sum_{x \in \D} \int_{\Y} \inner{\varphi(B y), \phi^\star(B y) - x} p(y \vert x) dy \notag \\
    &= \frac{1}{|\D|} \sum_{x \in \D} \int_{\Y} \inner{\varphi(B y), \phi^\star(B y) - x} \Normal{y; A x, \sigma^2 \Id} dy \label{eq:supp_linear_term}
\end{align}
This simplified expression (\ref{eq:supp_linear_term}) is useful for deriving the analytical solution of the MMSE estimator under various constraints.

\subsection{Unconstrained MMSE estimator}
We start with the unconstrained MMSE estimator, which is the optimal estimator in the least-square sense. Then, we will derive the equivariant MMSE estimator, which is the optimal estimator under the constraint of equivariance to translation. Then, we will derive the local MMSE estimator, which is the optimal estimator under the constraint of locality. Finally, we will also show that combining both constraints leads to a local and equivariant MMSE estimator.
\begin{proposition}[Unconstrained MMSE estimator]\label{prop:supp_unconstrained_mmse}
    When the data distribution is replaced by the empirical distribution $\ptrain$, the unconstrained MMSE estimator of $\vx$ given $\vy$ is given by
    \begin{equation}
        \xmmse(y) = \phi^\star(B y) \quad \text{where} \quad \phi^\star(v) = \frac{\sum_{x \in \D} x \cdot \Normal{v; B A x, \sigma^2 BB^\top}}{\sum_{x \in \D} \Normal{v; B A x, \sigma^2 BB^\top}} \quad \text{for any } v \in \Im{B}. 
    \end{equation}
    The estimator is well-defined for all $y \in \Y$.     
\end{proposition}
\begin{proof}
    We will verify that $\phi^*$ satisfies the optimality condition from \cref{prop:supp_optimality_condition}.
    Using \cref{eq:supp_linear_term}, for any $\varphi$, we have:
    \begin{align}
        \Mean{\inner{\varphi(B \y), \phi^*(B \y) - \x}} &=\frac{1}{|\D|}\sum_{x\in\D}\int_{\R^M} \langle \varphi(B y), \phi^*(B y) - x\rangle \Normal{y,Ax,\sigma^2 \Id_M} dy \notag\\
        &=\frac{1}{|\D|}\sum_{x\in\D}\int_{\R^N} \langle \varphi(v), \phi^*(v) - x \rangle \Normal{v,BAx,\sigma^2 BB^\top} d \Haus^r(v)\label{eq:proof_unconstrained_mmse_change_of_variable}\\
        &= \frac{1}{|\D|}\int_{\R^N} \left\langle \varphi(v), \phi^*(v) \sum_{x\in\D}\Normal{v,BAx,\sigma^2 BB^\top}-x \sum_{x\in\D}\Normal{v,BAx,\sigma^2 BB^\top}\right\rangle d \Haus^r(v) \notag\\
        &=0 \notag
    \end{align} 
    where \cref{eq:proof_unconstrained_mmse_change_of_variable} comes from the change of variable $v = B y$ and \cref{lemma:supp_integration_linear_subspace}.
    The last equality holds by definition of $\phi^*$. 
\end{proof}

\subsection{Equivariant MMSE estimator}\label{sec:supp_equivariant_mmse}

\begin{proposition}[Translation equivariant MMSE estimator]\label{prop:supp_equivariant_mmse}
    The translation equivariant MMSE estimator is as follows, for any $y \in \Y$:
        \begin{equation}
            \xtrans(y) = \phi^\star(B y) \qquad \text{where} \qquad \phi^\star(v) 
            =   \frac{\sum_{x \in \D, g \in \T} T_g x \cdot \Normal{T_g^{-1} v; B A x, \sigma^2 B B^\top}}{\sum_{x \in \D, g \in \T}  \Normal{T_g^{-1} v; B A x, \sigma^2 B B^\top}}
        \end{equation}
        for any $v \in \bigcup_{g} T_g \Im{B} \subset \X$.
\end{proposition}
\begin{proof}
    We will verify that $\phi^\star$ is admissible and satisfies the optimality condition \cref{prop:supp_optimality_condition}.

    \textbf{Admissibility.} 
    Let $q(v, x) = \Normal{v; B A x, \sigma^2 B B^\top}$ denote the PDF of a Gaussian distribution with mean $B A x$ and covariance $\sigma^2 B B^\top$. 
    The optimal estimator can be written as:
    \begin{equation}
        \phi^\star(v) = \frac{\sum_{x \in \D, g \in \T} T_g x q(T_g^{-1} v, x)}{\sum_{x \in \D, g \in \T}  q(T_g^{-1} v, x)}
    \end{equation}
    The denominator $\sum_{x \in \D, g \in \T}  q(T_g^{-1} v, x) = \sum_{g \in \T} \bar{q}_1(T_g^{-1} v)$ is $\T-$invariant by \cref{prop:supp_reynolds_averaging}, 
    where $\bar{q}_1(v) = \sum_{x \in \D} q(v, x)$.
    
    The nominator $\sum_{x \in \D, g \in \T} T_g x \cdot q(T_g^{-1} v, x) = \sum_{g \in \T} T_g \bar{q}_2(T_g^{-1} v)$ is $\T-$equivariant, where $\bar{q}_2(v) = \sum_{x \in \D} x \cdot q(v, x)$.
    Therefore, $\phi^\star$ is admissible, that is $\phi^\star \in \Mtrans$. 

    \textbf{Optimality condition.}
    By using \cref{eq:supp_linear_term}, for any $\varphi \in \Mtrans$, we have:
    \allowdisplaybreaks
    \begin{align}
        &\Mean{\inner{\varphi(B \y), \phi^\star(B  \y) - \vx}}  \notag \\
        &=  \frac{1}{|\D|}  \sum_{x \in \D} \int_{\Y} \inner{\varphi(B  y), \phi^\star(B  y) - x } \Normal{y; A x, \sigma^2 \Id_N} dy \notag \\
        &= \frac{1}{|\D|}  \sum_{x \in \D} \int_{\X} \inner{\varphi(v), \phi^\star(v) - x } \Normal{v; B A x, \sigma^2 B B^\top} d \Haus^r(v) \label{supp_proof:equiv_est_change_variable_operator} \\
        &= \frac{1}{|\D| |\T|} \sum_{g \in \T, x \in \D} \int_{\X} \inner{ T_g^{-1} \varphi(T_g v), T_g^{-1} \phi^\star( T_g v)
         - x } \Normal{v; B A x, \sigma^2 B B^\top} d \Haus^r(v)  \label{supp_proof:equiv_est_translation} \\
        &= \frac{1}{|\D| |\T|} \sum_{g \in \T, x\in \D} \int_{\X} \inner{\varphi(T_g v), \phi^\star(T_g v) - T_g x }  \Normal{v; B A x, \sigma^2 B B^\top} d\Haus^r(v) \notag \\
        &= \frac{1}{|\D| |\T|} \sum_{g \in \T, x\in \D} \int_{\X} \inner{\varphi(v), \phi^\star(v) - T_g x } \Normal{T_g^{-1} v; B A x, \sigma^2 B B^\top} d\Haus^r(v) \label{supp_proof:equiv_change_variable_translation} \\
        &= \frac{1}{|\D| |\T|} \int_{\X} \inner{\varphi(v), \phi^\star(v) \sum_{g \in \T, x\in \D} q(T_g^{-1} v, n, x) - \sum_{g \in \T, x\in \D} T_g x \cdot q(T_g^{-1} v, n, x)  } d\Haus^r(v) \notag \\
        &= 0 \notag
    \end{align}
    where $q(T_g^{-1} v, n, x) \eqdef  \Normal{T_g^{-1} v; B A x, \sigma^2 B B^\top}$ and
    \begin{itemize}
        \item In~\cref{supp_proof:equiv_est_change_variable_operator}, we apply \cref{lemma:supp_integration_linear_subspace} for $B$ in the change of variables. 
        \item In~\cref{supp_proof:equiv_est_translation}, we use the fact that $\varphi = \varphi \circ T_g \circ T_g^{-1} = T_g \circ \varphi \circ T_g^{-1}$ for all $g \in \T$ and $\varphi \in \Mtrans$. Moreover, $T_{g}^{-1} = T_g^{\top}$ for all $g \in \T$. 
        \item In~\cref{supp_proof:equiv_change_variable_translation}, we use the change of variable $T_g v \to v$ and the fact that $T_g$ is an isometry.
    \end{itemize} 
\end{proof}

We next state and prove that the augmented MMSE estimator is reconstruction equivariant.
    \begin{lemma}[Reconstruction equivariance of $\xmmseaug$]\label{lemma:supp_reconstruction_equivariance}
        Let $A: \R^N \to \R^M$ be a circular convolution operator, i.e., $A T_g = T_g A$ for all $g \in \T$.
        Then, the data-augmented MMSE estimator $\xmmseaug$ defined in\cref{def:data_augmented_mmse} satisfies the reconstruction equivariance property in~\cref{eq:reconstruction_equivariance}:
        \begin{equation*}
            \xmmseaug(A T_g \bar{x} + e) = T_g \xmmseaug(A \bar{x} + T_g^{-1} e)
        \end{equation*}
        for all $\bar{x} \in \R^N$, all $e \in \R^M$, and all $g \in \T$.
    \end{lemma}

    \begin{proof}
    Let $y = A T_g \bar{x} + e$ be the input observation.
    Recall the definition of the data-augmented MMSE estimator in~\cref{def:data_augmented_mmse}:
    \begin{equation*}
        \xmmseaug(y) = \frac{\sum_{x \in \T(\D)} x \cdot \exp\left( -\frac{1}{2\sigma^2} \norm{y - Ax}^2 \right)}{\sum_{x \in \T(\D)} \exp\left( -\frac{1}{2\sigma^2} \norm{y - Ax}^2 \right)}.
    \end{equation*}
    Substituting the input $y = A T_g \bar{x} + e$ into the estimator:
    \begin{equation*}
        \xmmseaug(y) = \frac{\sum_{x \in \T(\D)} x \cdot \exp\left( -\frac{1}{2\sigma^2} \norm{A T_g \bar{x} + e - Ax}^2 \right)}{\sum_{x \in \T(\D)} \exp\left( -\frac{1}{2\sigma^2} \norm{A T_g \bar{x} + e - Ax}^2 \right)}.
    \end{equation*}
    Since $\T(\D)$ is invariant under the group action, we can perform a change of variable $z = T_g^{-1} x$. As $x$ traverses $\T(\D)$, $z$ also traverses $\T(\D)$. Note that $x = T_g z$.
    Substituting $x$ with $T_g z$ in the numerator and denominator:
    \begin{equation*}
        \xmmseaug(y) = \frac{\sum_{z \in \T(\D)} T_g z \cdot \exp\left( -\frac{1}{2\sigma^2} \norm{A T_g \bar{x} + e - A T_g z}^2 \right)}{\sum_{z \in \T(\D)} \exp\left( -\frac{1}{2\sigma^2} \norm{A T_g \bar{x} + e - A T_g z}^2 \right)}.
    \end{equation*}
    Using the assumption that $A$ is a circular convolution ($A T_g = T_g A$) and $T_g$ is linear, the term inside the norm becomes:
    \begin{align*}
        A T_g \bar{x} + e - A T_g z &= T_g A \bar{x} + T_g (T_g^{-1} e) - T_g A z \\
        &= T_g \left( A \bar{x} + T_g^{-1} e - A z \right).
    \end{align*}
    Since $T_g$ is unitary, it preserves the norm, i.e., $\norm{T_g u} = \norm{u}$. Therefore:
    \begin{equation*}
        \norm{A T_g \bar{x} + e - A T_g z}^2 = \norm{A \bar{x} + T_g^{-1} e - A z}^2.
    \end{equation*}
    Substituting this back into the estimator expression and factoring the linear operator $T_g$ out of the sum in the numerator:
    \begin{align*}
        \xmmseaug(y) &= \frac{T_g \sum_{z \in \T(\D)} z \cdot \exp\left( -\frac{1}{2\sigma^2} \norm{(A \bar{x} + T_g^{-1} e) - A z}^2 \right)}{\sum_{z \in \T(\D)} \exp\left( -\frac{1}{2\sigma^2} \norm{(A \bar{x} + T_g^{-1} e) - A z}^2 \right)} \\
        &= T_g \left( \frac{\sum_{z \in \T(\D)} z \cdot \Normal{A \bar{x} + T_g^{-1} e; A z, \sigma^2 \Id}}{\sum_{z' \in \T(\D)} \Normal{A \bar{x} + T_g^{-1} e; A z', \sigma^2 \Id}} \right).
    \end{align*}
    The term in the parentheses is exactly the estimator evaluated at the input $A \bar{x} + T_g^{-1} e$. Thus:
    \begin{equation*}
        \xmmseaug(A T_g \bar{x} + e) = T_g \xmmseaug(A \bar{x} + T_g^{-1} e).
    \end{equation*}
    \end{proof}

Now we state and prove the properties of the E-MMSE estimator presented in~\cref{cor:equivariant_mmse_properties}.

\begin{proof}[Proof of~\cref{cor:equivariant_mmse_properties}]\label{proof:supp_equivariant_properties}
    The first point is a direct consequence of~\cref{thm:equivariant_mmse}.  
    We prove the second and third point as follows.
    We first express the weights of both estimators.
    The two estimators admit the same form:
    \begin{equation*}
        \xtrans(y) = \sum_{x \in \D, g \in \T} T_g x \cdot w_g(x \vert y) \qquad \text{and} \qquad \xmmseaug(y) = \sum_{x \in \D, g \in \T} T_g x \cdot w_g^{\text{aug}}(x \vert y).
    \end{equation*}
    The weights of the E-MMSE estimator $\xtrans$ corresponding to the component $(x, g)$ are given by:
    \begin{align*}
        w_g(x \vert y) 
        &= \frac{\Normal{T_g^{-1} B y; B A x, \sigma^2 B B^{\top}}}{\sum_{x' \in \D, g' \in \T} \Normal{T_{g'}^{-1} B y; B A x', \sigma^2 B B^{\top}}}
        \\&= \frac{\exp \left( -\norm{B^{-1} (T_g^{-1} B y - B A x)}^2 / (2 \sigma^2) \right)}{\sum_{x' \in \D, g' \in \T} \exp \left( - \norm{B^{-1} (T_{g'}^{-1} B y - B A x')}^2 / (2 \sigma^2) \right)}
        \\&= \frac{\exp \left( -\norm{B^{-1} T_g^{-1} B y - A x}^2 / (2 \sigma^2) \right)}{\sum_{x' \in \D, g' \in \T} \exp \left( -\norm{B^{-1} T_{g'}^{-1} B y - A x'}^2 / (2 \sigma^2) \right)}
    \end{align*}
    where we used the invertibility of $B$ to write $B^+ = B^{-1}$ and simplified the Mahalanobis distance.
    The weights of the data-augmented estimator $\xmmseaug$ are:
    \begin{align*}
        w_g^{\text{aug}}(x \vert y) 
        &= \frac{\Normal{B y; B A T_g x, \sigma^2 B B^\top}}{\sum_{x' \in \D, g' \in \T} \Normal{B y; B A T_{g'} x', \sigma^2 B B^\top}}
        \\&= \frac{\exp \left( -\norm{B^{-1}(B y - B A T_g x)}^2 / (2 \sigma^2) \right)}{\sum_{x' \in \D, g' \in \T} \exp \left( -\norm{B^{-1}(B y - B A T_{g'} x')}^2 / (2 \sigma^2) \right)}
        \\&= \frac{\exp \left( -\norm{y - A T_g x}^2 / (2 \sigma^2) \right)}{\sum_{x' \in \D, g' \in \T} \exp \left( -\norm{y - A T_{g'} x'}^2 / (2 \sigma^2) \right)}.
    \end{align*}
    
    \paragraph{Sufficiency ($\Leftarrow$)}
    Assume $A$ and $B$ are circular convolutions (they commute with $T_g$) and $B$ is invertible.

    Commutativity implies $B^{-1} T_g^{-1} = T_g^{-1} B^{-1}$. We can rearrange the term in the exponent of the above weights as follows:
    \begin{equation*}
        \norm{B^{-1} T_g^{-1} B y - A x}^2
        = \norm{B^{-1} B T_g^{-1} y - B^{-1} B A x}^2
        = \norm{T_g^{-1} y - A x}^2.
    \end{equation*}
    Since $T_g$ is unitary and $A$ commutes with $T_g$, we have:
    \begin{equation*}
        \norm{T_g^{-1} y - A x}^2 = \norm{T_g(T_g^{-1} y - A x)}^2 = \norm{y - T_g A x}^2 = \norm{y - A T_g x}^2.
    \end{equation*}
    Thus $\xtrans = \xmmseaug$. 
    Furthermore, since the weights are independent of $B$, the physics-agnostic and physics-aware solvers are identical. 
    The reconstruction equivariance follows immediately from~\cref{lemma:supp_reconstruction_equivariance}.

\paragraph{Necessity ($\Rightarrow$)}
Assume that  $w_g^{\text{aug}}(x \vert y) =  w_g(x \vert y)$ for all $y \in \R^M$ and all $x\in \R^N$ and that $A$ and $B$ are invertible.
This implies that:
\begin{equation}\label{eq:nec_energy_equality}
    \norm{B^{-1} T_g^{-1} B y - A x}^2 = \norm{y - A T_g x}^2 + c(y)
\end{equation}
for some function $c(y)$ independent of $x$ and $g$. We expand the squared norms and the terms depending solely on $y$ can be absorbed into $c(y)$.
\begin{align*}
    \text{LHS}
    &= \underbrace{\norm{B^{-1} T_g^{-1} B y}^2}_{\text{depends only on } y} - 2 \langle B^{-1} T_g^{-1} B y, A x \rangle + \norm{A x}^2, \\
    \text{RHS} &= \underbrace{\norm{y}^2}_{\text{depends only on } y} - 2 \langle y, A T_g x \rangle + \norm{A T_g x}^2 + c(y).
\end{align*}
For the equality in \cref{eq:nec_energy_equality} to hold for all $x\in \R^N$ and $y\in \R^M$, the cross-terms (the bilinear forms coupling $y$ and $x$) must be identical.
Moreover, since the estimator $\xmmseaug$ is \emph{independent} of $B$, the~\cref{eq:nec_energy_equality} must hold for all invertible $B$.
In particular, taking $B = \Id$, we deduce that
\begin{equation*}
    \inner{T_g^{-1}y, Ax} = \inner{y, A T_g x} \quad \text{for all } x \in \R^N, g \in \T, y \in \R^M.
\end{equation*}
This equality yields $T_g A = A T_g$, or that $A$ commutes with $T_g$.
Plugging this result into the cross-term equality for a general invertible $B$, we get:
\begin{equation*}
    \inner{B^{-1} T_g^{-1} B y, A x} = \inner{y, A T_g x} = \inner{T_g^{-1} y, A x} \quad \text{for all } x \in \R^N, g \in \T, y \in \R^M.
\end{equation*}
This implies that $A^\top B^{-1} T_g^{-1} B = A^\top T_g^{-1}$ for all $g$. 
Since $A$ is invertible, this implies that $B^{-1} T_g^{-1} B = T_g^{-1}$ for all $g\in \T$.  
Multiplying by $B$ on each side of the equality implies that $B$ commutes with $T_g$, concluding the proof.
\end{proof}


\subsection{Local and Translation Equivariant MMSE estimator}\label{sec:supp_local_trans_mmse}
\begin{proposition}[Local and translation equivariant MMSE estimator]
    \label{prop:supp_local_trans_estimator}
        Suppose that the $N$ matrices $Q_{n} = \Pi_n B \in \R^{P \times M}$ have \textbf{constant rank} $r>0$.
        The local and equivariant MMSE is defined for any $y \in \Y$ by:  
        \begin{equation}
            \xtransloc(y) = \phi^\star(B y),
        \end{equation}
        where
        \begin{equation}
            \phi^\star_n = f_{\phi^\star} \circ \Pi_n \qquad \text{with} \qquad
            f_{\phi^\star}(v) = \frac{\sum_{x \in \D} \sum_{n = 1}^{N} x_n \Normal{v; Q_n A x, \sigma^2 Q_n Q_n^\top}}{\sum_{x \in \D} \sum_{n = 1}^{N} \Normal{v; Q_n A x, \sigma^2 Q_n Q_n^\top}}.
        \end{equation}
		% {\color{blue}(there is no $v$ in the formula for $f$, I guess this is $z$. Also in the proof, this could be harmonized as $z$ turn to $v$ sometimes.)}
\end{proposition}
\begin{proof}
    We will verify that $\phi^\star$ is admissible and satisfies the optimality condition \cref{prop:supp_optimality_condition}.

    \textbf{Admissibility.}
    Firstly, we show that $\phi^\star \in \Mtransloc$. By construction, we have $\phi^\star = (\phi^\star_1, \cdots \phi^\star_N) = (f_{\phi^\star} \circ \Pi_1, \cdots f_{\phi^\star} \circ \Pi_N)$, therefore $\phi^{\star}$ is local. 
    For any $g \in \left\{ 1, \cdots N \right\}$, the group transformation (translation) $T_g$ is defined as 
    $T_g : (x_1, \cdots, x_N) \in \X \mapsto (x_{1 - g}, \cdots, x_{N - g}) \in \X$, where $x_{n - g} = x_{n - g \equiv N}$ (circular boundary). For any $g$, we have 
    \begin{align*}
        \phi^\star \circ T_g &=  (\phi^\star_1 \circ T_g , \cdots \phi^\star_N \circ T_g) \\
        &= (f_{\phi^\star} \circ \Pi_1 \circ T_g, \cdots f_{\phi^\star} \circ \Pi_N \circ T_g)\\
        &= (f_{\phi^\star} \circ \Pi_{1 - g}, \cdots f_{\phi^\star} \circ \Pi_{N - g})  \qquad \text{ where } \Pi_{n - g} = \Pi_{n - g \equiv N} \text{ (circular boundary)}\\
        &= T_g \circ \phi^\star 
    \end{align*}
    Therefore, $\phi^{\star}$ is translation equivariant. We deduce that $\phi^\star \in \Mtransloc$. 

    \textbf{Optimality condition.} We will verify that this estimator satisfies the optimality condition \cref{prop:supp_optimality_condition}.
    By using \cref{eq:supp_linear_term}, for any $\varphi \in \Mtransloc$, we have:
    \allowdisplaybreaks
    \begin{align*}
        &\Mean{\inner{\varphi(B \y), \phi^\star(B  \y) - \x}}  \\
        &= \frac{1}{|\D|}  \sum_{x \in \D} \int_{\Y} \inner{\varphi(B  y), \phi^\star(B  y) - x } \Normal{y; A x, \sigma^2 \Id_N} dy \notag \\
        &= \frac{1}{|\D|}  \sum_{x \in \D} \sum_{n = 1}^{N} \int_{\Y} f_{\varphi} (\Pi_n B  y) \left( f_{\phi^\star} (\Pi_n B y) - x_n  \right) \Normal{y; A x, \sigma^2 \Id_N} dy \notag \\
        &= \frac{1}{|\D|}  \sum_{x \in \D} \sum_{n = 1}^{N} \int_{\R^P} f_{\varphi} (v) \left( f_{\phi^\star} (v) - x_n  \right) \underbrace{\Normal{v; Q_{n} A x, \sigma^2 Q_{n} Q_{n}^\top}}_{q(v, n, x)} d \Haus^r(v) \notag \\
        &= \frac{1}{|\D|} \int_{\R^P} f_{\varphi} (v) \left( f_{\phi^\star} (v) \sum_{x \in \D} \sum_{n = 1}^{N}  q(v, n, x) - \sum_{x \in \D} \sum_{n = 1}^{N} x_n q(v, n, x) \right) d \Haus^r(v) \notag \\
        &= 0
    \end{align*}
\end{proof}

The constant rank assumption in~\cref{prop:supp_local_trans_estimator} can be relaxed by stratifying the image space according to the rank of the matrices $Q_n$ as follows.
\begin{proposition}[Local and translation equivariant MMSE estimator -- rank stratification]
    \label{prop:supp_local_trans_estimator_general}
    Let $Q_n = \Pi_n B \in \R^{P \times M}$. For each rank $r \in \{0,1,\dots,P\}$, define the index set and the union of subspaces:
    $$
        I_r = \{ n \in [1\!:\!N] : \rank{Q_n} = r\}, \quad 
        E_r = \bigcup_{n \in I_r} \Im{Q_n} \subset \R^P.
    $$
    We define the disjoint strata recursively: $\bar{E}_r = E_r \setminus \bigcup_{r' < r} E_{r'}$.
    Then $\bigcup_{r=0}^{P} \bar{E}_r = \bigcup_{n=1}^{N} \Im{Q_n}$ is a disjoint union and, for any $r' < r$, we have $\Haus^r(E_{r'}) = 0$.
    
    Define the weighted density sums for $v \in \R^P$:
    \begin{align*}
        a_r(v) &= \sum_{x \in \D} \sum_{n \in I_r} x_n \Normal{v; Q_n A x, \sigma^2 Q_n Q_n^\top}, \\
        b_r(v) &= \sum_{x \in \D} \sum_{n \in I_r} \Normal{v; Q_n A x, \sigma^2 Q_n Q_n^\top}.
    \end{align*}
    The local and translation equivariant MMSE estimator $\phi^\star$ is given component-wise by $\phi^\star_n = f_{\phi^\star} \circ \Pi_n$, where:
    \begin{equation}
        f_{\phi^\star}(v) = 
        \begin{cases}
             \sum_{r=0}^{P} \frac{a_r(v)}{b_r(v)} \mathbbm{1}_{\bar{E}_r}(v) & \text{if } v \in \bigcup \Im Q_n \text{ and } b_r(v) > 0, \\
             0 & \text{otherwise}.
        \end{cases}
    \end{equation}
\end{proposition}

\begin{proof}
    \textbf{Admissibility (locality and translation equivariance).}
    \begin{itemize}
        \item Locality. By definition, the estimator is defined component-wise by $\phi^\star_n = f_{\phi^\star} \circ \Pi_n$, so it is local.

        \item Translation equivariance on $\Im B$. Let $T_g$ denote the circular translation on $\X$, the selection operators commute with $T_g$ by a simple index check 
        $$ \Pi_n \circ T_g = \Pi_{n-g}, \qquad \forall\,n,g\in \llbracket 1, N \rrbracket.$$
        Then, for any $v\in \Im B$ and any $n$,
        \begin{equation}
            \big[\phi^\star(T_g v)\big]_n
            = f_{\phi^\star}\big(\Pi_n T_g v\big) 
            = f_{\phi^\star}\big(\Pi_{n-g} v\big)
            = \big[\phi^\star(v)\big]_{n-g}
            = \big[T_g \phi^\star(v)\big]_n.
        \end{equation}
        Hence $\phi^\star \circ T_g = T_g \circ \phi^\star$ on $\Im B$, i.e., it is translation equivariant. Combining with locality gives $\phi^\star \in \Mtransloc$.
        
        \item Well-definedness. On each $\bar{E}_r$, $f_{\phi^\star}$ is defined by the ratio $a_r/b_r$. 
        If $b_r(v)=0$ on a negligible set (w.r.t. $\Haus^r$), assign any fixed value (\eg $0$); this does not affect admissibility nor optimality.
    \end{itemize}

    \textbf{Optimality.} For any $\varphi \in \Mtransloc$, we will verify that the optimality condition \cref{prop:supp_optimality_condition} holds.
    By using \cref{eq:supp_linear_term}, we have:
    \begin{align*}
        &\Mean{\inner{\varphi(B \vy), \phi^\star(B \vy) - \vx}} \\
        &= \frac{1}{|\D|}  \sum_{x \in \D} \int_{\Y} \inner{\varphi(B  y), \phi^\star(B  y) - x } \Normal{y; A x, \sigma^2 \Id_N} dy \notag \\
        &= \frac{1}{|\D|} \sum_{x \in \D} \sum_{n=1}^{N} \int_{\Y} f_{\varphi}(\Pi_n B y) \big( f_{\phi^\star}(\Pi_n B y) - x_n \big) \Normal{y; A x, \sigma^2 I} \, dy \\
        &\overset{(\star)}{=} \frac{1}{|\D|} \sum_{x \in \D} \sum_{n=1}^{N} \int_{\R^P} f_{\varphi}(v) \big( f_{\phi^\star}(v) - x_n \big) \Normal{v; Q_n A x, \sigma^2 Q_n Q_n^\top} \, d\Haus^{r_n}(v),
    \end{align*}
    where $(\star)$ uses \cref{lemma:supp_integration_linear_subspace} and $r_n = \rank{Q_n}$.
    We decompose the summation over $n$ by rank. Note that for $n \in I_r$, the domain is $\Im Q_n$. We observe that $\Im Q_n \setminus \bar{E}_r \subseteq \bigcup_{r' < r} E_{r'}$. Since the union of lower-dimensional subspaces has $\Haus^r$-measure zero, we can restrict the integration domain from $\Im Q_n$ to $\Im Q_n \cap \bar{E}_r$ without changing the value of the integral. 
    Thus, the previous expression becomes: 
    \begin{align}
        \sum_{r=0}^{P} &\int_{\bar{E}_r}
            f_{\varphi}(v) \Big( f_{\phi^\star}(v) \underbrace{\sum_{x \in \D} \sum_{n \in I_r} \Normal{v; Q_n A x, \sigma^2 Q_n Q_n^\top}}_{b_r(v)}
            \\& \hspace{4cm}
            - \underbrace{\sum_{x \in \D} \sum_{n \in I_r} x_n \Normal{v; Q_n A x, \sigma^2 Q_n Q_n^\top}}_{a_r(v)} \Big) d\Haus^{r}(v).
    \end{align}
    Choosing $f_{\phi^\star}(v)=a_r(v)/b_r(v)$ on $\bar{E}_r$ cancels each integrand, hence the optimality condition \cref{prop:supp_optimality_condition} holds.
\end{proof}

\begin{remark}
    If all $Q_n$ have the same rank $r$, then $\bar{E}_r = \bigcup_{n=1}^{N} \Im{Q_n}$ and the formula reduces to the constant-rank case in \cref{prop:supp_local_trans_estimator}, with a single ratio over $n=1,\dots,N$.
\end{remark}

\begin{remark}[Singular limit and rank stratification]\label{remark:rank_deficient_B}
    Consider a regularization of $B$ as $B^{(\epsilon)} = U \Sigma_\epsilon V^\top$ and $Q_n^{(\epsilon)} = \Pi_n B^{(\epsilon)}$, where $B = U \Sigma V^\top$ is the SVD of $B$ 
    and the diagonal matrix $\Sigma_\epsilon$ is constructed by adding a small positive number $\epsilon$ to the singular values in $\Sigma$.  
    For $\epsilon>0$ (and $M\ge P$), each $Q_n^{(\epsilon)}$ has full row rank $P$. For each $(n,x)$, define the probability measure $\mu_{n,x}^{(\epsilon)}$ on $\R^P$ with density (Radon--Nikodým derivative) with respect to Lebesgue measure $\lambda^P$ given by
    $$
        \frac{d\mu_{n,x}^{(\epsilon)}}{d\lambda^P}(v)
        \;=\; \Normal{v; Q_n^{(\epsilon)} A x, 
        \sigma^2 Q_n^{(\epsilon)} Q_n^{(\epsilon)T}}.
    $$
    As $\epsilon \to 0$, $Q_n^{(\epsilon)} \to Q_n$ and some ranks $r_n = \rank Q_n$ may drop. Then $\mu_{n,x}^{(\epsilon)}$ converges weakly to a probability measure $\mu_{n,x}$ supported on the linear subspace $\Im Q_n$, which is absolutely continuous with respect to the Hausdorff measure $\Haus^{r_n}$ on $\Im Q_n$, with density
    $$
        \frac{d\mu_{n,x}}{d\Haus^{r_n}}(v)
        \;=\; \Normal{v; Q_n A x, \sigma^2 Q_n Q_n^\top},
        \qquad v \in \Im{Q_n},
    $$
    where, $\Normal{\cdot;\mu,\Sigma}$ denotes the degenerate Gaussian density~\cref{def:supp_degenerate_gaussian} with respect to the appropriate Hausdorff measure on its support (and vanishes off that support).

    For $\epsilon>0$, the estimator reads (constant-rank case, similar to \cref{prop:supp_local_trans_estimator})
    $$
        f_{\phi^\star}^{(\epsilon)}(v)
        \;=\; \frac{\sum_{x \in \D} \sum_{n=1}^{N} x_n \, \Normal{v; Q_n A x, \sigma^2 Q_n Q_n^\top}}{\sum_{x \in \D} \sum_{n=1}^{N} \Normal{v; Q_n A x, \sigma^2 Q_n Q_n^\top}}.
    $$
    In the singular limit, for each $r$ and $\Haus^r$-a.e. $v \in \bar{E}_r$ with $b_r(v)>0$,
    $$
        \lim_{\epsilon\to 0} f_{\phi^\star}^{(\epsilon)}(v)
        \;=\; \frac{a_r(v)}{b_r(v)},
    $$
    \ie the $\epsilon$-regularized estimator converges (stratum-wise, $\Haus^r$-a.e.) to the rank-stratified formula
    $$
        f_{\phi^\star}(v) 
        \;=\; \sum_{r=0}^{P} \frac{a_r(v)}{b_r(v)}\, \mathbbm{1}_{\bar{E}_r}(v),
    $$
    interpreted up to $\Haus^r$-null sets on each $\bar{E}_r$. If all $Q_n$ have the same rank $r$, only the stratum $\bar{E}_r$ is nonempty, and the above reduces to the constant-rank expression in ~\cref{prop:supp_local_trans_estimator}.
\end{remark}

\begin{proof}[Proof of i) of~\cref{cor:local_trans_estimator_properties} --- Physics-agnostic LE-MMSE estimator]
    When $B = \Id$, the weights of the LE-MMSE estimator in~\cref{theorem:local_trans_estimator} simplifies to 
    \begin{equation*}
        w_{n', n}(x \vert y) \propto \Normal{\Pi_{n'} y; \Pi_{n} A x, \sigma^2 \Pi_{n} \Pi_{n}^\top} \propto \exp \left( - \frac{1}{2\sigma^2} \norm{y[\omega_{n'}] - (A x)[\omega_{n}]}^2 \right)
    \end{equation*}
    Therefore, the LE-MMSE estimator at pixel $n'$ reads:
    \begin{equation*}
        \xtransloc(y)[n'] = \frac{\sum_{x \in \D} \sum_{n = 1}^{N} x_n \exp \left( - \frac{1}{2\sigma^2} \norm{y[\omega_{n'}] - (A x)[\omega_{n}]}^2 \right)}{\sum_{x \in \D} \sum_{n = 1}^{N} \exp \left( - \frac{1}{2\sigma^2} \norm{y[\omega_{n'}] - (A x)[\omega_{n}]}^2 \right)}.
    \end{equation*}
    For small noise level $\sigma \to 0$, it returns the central pixel value of the patches in the dataset $\D$ whose degraded version $(A x)[\omega_n]$ is closest to the observed patch $y[\omega_{n'}]$. Hence, the LE-MMSE estimator is a patch-work of training patches.
\end{proof}
\begin{proof}[Proof of ii) of~\cref{cor:local_trans_estimator_properties} --- LE-MMSE is not a posterior mean]
    We focus on the simplest case of denoising, that is $\vy = \vx + \ve$.
    Recall that the posterior mean satisfies the so-called Tweedie formula:
    \begin{equation*}
        \Mean{\vx \vert y} = y + \sigma^2 \nabla \log p_{\vy}(y).
    \end{equation*}
    Taking the gradient of both sides w.r.t $y$ yields:
    \begin{equation*}
        \nabla_y \Mean{\vx \vert y} = \Id + \sigma^2 \nabla^2 \log p_{\vy}(y).
    \end{equation*}
    Therefore, the Jacobian of any pure MMSE estimator (posterior mean) must be symmetric since the Hessian of $\log p_{\vy}$ is symmetric. 
    To simplify the notation, we use $f$ instead of $\xtransloc$ in what follows.
    Using the fact that the scaling by $\sigma^{-2}$ does not affect the symmetry (note that $p_{\vy}$ is the convolution of $p_{\vx}$ and a Gaussian so it's $\mathcal{C}^{\infty}$), if $\xtransloc$ were a posterior mean, we must have:
    \begin{equation}\label{eq:supp_posterior_mean_condition}
        \frac{\partial}{\partial y_m} f_{n}(y) = \frac{\partial}{\partial y_{n}} f_m(y), \quad \forall n,m \text{ and } \forall y.
    \end{equation}    
    For notation simplicity, we let $e_{n, l}(x, y) = \exp \left( - \frac{\|y[\omega_{n}] - x[\omega_l]\|^2}{2\sigma^2}\right)$.
    We have the weights of the LE-MMSE estimator in~\cref{theorem:local_trans_estimator} becomes:
    \begin{equation*}
        w_{n, l} (x \vert y) = \exp \left( - \frac{\norm{y[\omega_{n}] - x[\omega_l]}}{2\sigma^2}\right) / Z_{n}(y) = \frac{e_{n, l}(x, y)}{Z_n(y)}.
    \end{equation*}
    where $Z_n(y) = \sum_{x' \in \D} \sum_{l'=1}^N e_{n, l'}(x', y)$ is the normalization constant.
    Therefore, the LE-MMSE estimator at pixel $n$ reads:
    \begin{equation*}
        f_{n}(y) = \frac{1}{Z_{n}(y)} \sum_{x\in \D} \sum_{l=1}^N x_l \cdot e_{n, l}(x, y).
    \end{equation*}
    That is:
    \begin{equation*}
        f_{n}(y) = \frac{S_{n}(y)}{Z_{n}(y)} \qquad \text{where} \qquad S_{n}(y) = \sum_{x\in \D} \sum_{l=1}^N x_l \cdot e_{n, l}(x, y).
    \end{equation*}
    A key observation is that \emph{this estimator is real-analytic for $\sigma>0$}. 
    To prove it, we remind that the product of polynomials and exponentials are real-analytic functions, and that the sum and quotient (with non-vanishing denominator) of real-analytic functions are also real-analytic.
    The denominator $Z_n(y)$ does not vanish for $\sigma>0$, which concludes the proof of real-analyticity.

	We will use the following classical result (see e.g. \citep[Section 3.1.24]{federer1996geometric} and \citep{mityagin2020zero} for an elementary proof):
    \begin{lemma}[Zeros of real-analytic functions]\label{lemma:zero_measure_analytic_eq}
		Let $f : \R^{p} \to \R$ be a real-analytic function for some $p \in \mathbb{N}^*$. If $f$ is not identically zero, then the set of zeros of $f$ has Lebesgue measure zero.
    \end{lemma}

    Applying this lemma, it therefore suffices to find a dataset $\D$ and a point $y$ such that $\frac{\partial}{\partial y_m} f_{n}(y) - \frac{\partial}{\partial y_{n}} f_m(y) \neq 0$.
    To this end, we consider the case of overlapping patches $\omega_n\cap \omega_m \neq \emptyset$ with $n \neq m$, which is always possible for patch sizes $P \geq 2$.
    Now consider the case where $n \in \omega_{m}$ and $m \in \omega_{n}$ (they are equivalent). 

    The chain rule gives 
    \[
        \frac{\partial e_{n, l}}{\partial y_m}(x, y) = -\sigma^{-2} (y_m - x_{l + m - n}) \cdot e_{n, l}(x, y) 
    \]
    and the quotient rule gives: 
    \begin{align*}
        &\frac{\partial w_{n, l}}{\partial y_m}(x, y) = \frac{\partial e_{n, l}}{\partial y_m}(x, y) \cdot \frac{1}{Z_{n}(y)} - e_{n, l}(x, y) \cdot \frac{\partial Z_{n}}{\partial y_m}(y) \cdot \frac{1}{Z_{n}(y)^2} \\
        &= -\sigma^{-2} (y_m - x_{l + m - n}) \cdot \frac{e_{n, l}(x, y)}{Z_{n}(y)} -  \frac{e_{n, l}(x, y)}{Z_{n}(y)^2} \cdot \frac{\partial Z_{n}(y)}{\partial y_m} \\
        &= -\sigma^{-2} (y_m - x_{l + m - n}) \cdot w_{n, l}(x, y) -  w_{n, l}(x, y) \cdot \frac{1}{Z_{n}(y)} \cdot \left( \sum_{x' \in \D} \sum_{l'=1}^{N} \frac{\partial e_{n, l'}}{\partial y_m}(x',y) \right) \\ 
        &= -\sigma^{-2} \left( (y_m - x_{l + m - n}) \cdot w_{n, l}(x, y) +  w_{n, l}(x, y) \cdot \frac{1}{Z_{n}(y)} \cdot \left( \sum_{x' \in \D} \sum_{l'=1}^{N} (y_m - x_{l' + m - n}) \cdot e_{n, l'}(x',y) \right) \right)\\
        &= -\sigma^{-2} \left((y_m - x_{l + m - n}) \cdot w_{n, l}(x, y) +  w_{n, l}(x, y) \cdot \frac{1}{Z_{n}(y)} \cdot \left( y_m \cdot Z_{n}(y) - \sum_{x' \in \D} \sum_{l'=1}^{N} x_{l' + m - n} \cdot e_{n, l'}(x',y) \right)\right) \\
        &=  \sigma^{-2} \cdot \left( x_{l + m - n} \cdot w_{n, l}(x, y) - w_{n, l}(x, y) \cdot \frac{1}{Z_{n}(y)} \cdot \sum_{x' \in \D} \sum_{l'=1}^{N} x_{l' + m - n} \cdot e_{n, l'}(x',y) \right)\\
        &=  \sigma^{-2} \cdot \left( x_{l + m - n} \cdot w_{n, l}(x, y) - w_{n, l}(x, y) \cdot \sum_{x' \in \D} \sum_{l'=1}^{N} x_{l' + m - n} \cdot w_{n, l'}(x',y) \right)\\
        &= \sigma^{-2} w_{n, l}(x, y)  \left( x_{l + m - n} - \bar{x}_n^{(m - n)}(y)  \right)
    \end{align*}
    where we define $\bar{x}_n^{(k)}(y) = \sum_{x' \in \D} \sum_{l'=1}^{N} x_{l' + k} \cdot w_{n, l'}(x',y)$ the local posterior mean of pixel $n$ with offset $k$. 
    Finally, differentiating $f_n(y)$ gives:
    \begin{align*}
        \frac{\partial f_{n}}{\partial y_m}(y) &= \sum_{x \in \D} \sum_{l=1}^N x_l \cdot \frac{\partial w_{n, l}}{\partial y_m}(x, y) \\
        &= \sigma^{-2} \sum_{x \in \D} \sum_{l=1}^N x_l \cdot w_{n, l}(x, y)  \left( x_{l + m - n} - \bar{x}_n^{(m - n)}(y)  \right) \\
        % &= \sigma^{-2} \sum_{x \in \D} \sum_{l=1}^N x_l \cdot x_{l + m - n} \cdot w_{n, l}(x, y) - \sigma^{-2} f_{n}(y) \cdot \bar{x}_n^{(m - n)}(y)
    \end{align*}

	Set $n = 2$ and $m=3$. 
    Set $\D = \{x, 0, \dots 0\}$ to be dataset containing only one non-zero image $x$ and the rest are zeros.
    We choose $x$ such that $x_2 > 0$ and $x_i = 0$ for $i \neq 2$.
	The derivatives simplify to:
	\begin{align*}
		\frac{\partial f_2}{\partial y_3}(y) &= \sigma^{-2}x_2 w_{2,2}(x, y) (x_3  - \bar{x}_2^{(1)}(y)) = -\sigma^{-2}x_2^2 w_{2,2}(x, y) w_{2,1}(x,y) \\
		\frac{\partial f_3}{\partial y_2}(y) &= \sigma^{-2} x_2 w_{3,2}(x, y) (x_1  - \bar{x}_3^{(-1)}(y)) = -\sigma^{-2}x_2^2 w_{3,2}(x,y)w_{3,3}(x,y)
	\end{align*}

	Fix an index $1 \leq i \leq N$ such that $y_i$ is a variable appearing in the vector $y[\omega_3]$ but not in the vector $y[\omega_2]$ (note that $i \neq 3$, otherwise $y_i = y_3$ is the center of the patch $y[\omega_3]$ and it's also in the patch $y[\omega_2]$ since the patch size $P > 1$). We have
	\begin{align*}
		\frac{\partial w_{2,2}(x,y)}{\partial y_i} = \frac{\partial w_{2,1}(x,y)}{\partial y_i} = 0 \quad \text{and} \quad \frac{\partial^2 f_2}{\partial y_3 y_i}(x,y) = 0.
	\end{align*}
	On the other hand, since $ i \neq 3$, we have
	\begin{align*}
		\frac{\partial w_{3,2}(x,y)}{\partial y_i} &= \sigma^{-2} w_{3, 2}(x, y)  \left( x_{2 + i - 3} - \bar{x}_3^{(i - 3)}(y)  \right) = -\sigma^{-2} x_2 w_{3, 2}(x, y) w_{3,2 - (i-3)}(x,y) \\
		\frac{\partial w_{3,3}(x,y)}{\partial y_i} &= \sigma^{-2} w_{3, 3}(x, y)  \left( x_{i } - \bar{x}_3^{(i - 3)}(y)  \right) = -\sigma^{-2} x_2 w_{3, 3}(x, y) w_{3,2 - (i-3)}(x,y). 
	\end{align*}
	We obtain
	\begin{align*}
		\frac{\partial^2 f_3(x,y)}{\partial y_2 y_i} &=\sigma^{-4} x_2^3 w_{3, 2}(x, y) w_{3,2 - (i-3)}w_{3,3}(x,y) +\sigma^{-4} x_2^3 w_{3, 3}(x, y) w_{3,2 - (i-3)}(x,y)w_{3,2} (x,y) > 0.
	\end{align*}
	Therefore, for this value of $x$, $\frac{\partial f_3}{\partial y_2}$ and $\frac{\partial f_2}{\partial y_3}$ are independent, when seen as functions of $y$. 
	We deduce that the equation $\frac{\partial f_2}{\partial y_3} = \frac{\partial f_3}{\partial y_2}$ 
	is a nondegenerate analytic equation in variables $\mathcal{D},y$. Using~\cref{lemma:zero_measure_analytic_eq} and by Fubini's theorem, the set $\left\{ \mathcal{D},\, \forall y,\, \frac{\partial f_2}{\partial y_3}(y) = \frac{\partial f_3}{\partial y_2}(y)\right\}$ also has zero measure.

    In other words, for almost every dataset $\D$, there exists $y$ such that the symmetry condition~\cref{eq:supp_posterior_mean_condition} does not hold, and the LE-MMSE estimator is not a posterior mean.
\end{proof}
% ###########################################################
% ###########################################################
% ###########################################################
% ###########################################################
% ###########################################################
% ###########################################################


% ###########################################################
%           NUMERICAL
% ########################################################### 
\null 
\pagebreak
\section{Numerical experiments}\label{sec:supp_numerical}
\subsection{Datasets}
We use the following datasets in our experiments: FFHQ~\citep{karras2019style} downscaled to $32 \times 32$ or $64 \times 64$, CIFAR-10~\citep{krizhevsky2009learning} and Fashion-MNIST~\citep{xiao2017fashion}.
For each dataset, we randomly select $10,000$ images for training, which is denoted by $\D$ in the main paper. 
\subsection{Network architectures}\label{sec:supp_network_architectures}
For the local and translation equivariant estimator, we examine 3 different architectures with noise level $\sigma$ conditioning:
\begin{itemize}
    \item UNet2D: we use a state-of-the-art UNet2D~\citep{ronneberger2015u} from the \texttt{diffusers}\footnote{\href{https://huggingface.co/docs/diffusers/en/api/models/unet2d}{https://huggingface.co/docs/diffusers/en/api/models/unet2d}} library. 
    The model has several downsampling and upsampling layers and skip connections, with varying channel dimensions (defined by \texttt{block\_out\_channels}) and kernel size (defined by \texttt{kernel\_size}, for down-blocks and mid-blocks). 
    The architecture is modified slightly (circular boundary conditions and kernel sizes of convolutional layers) to have a desired receptive field and to ensure translation equivariance. 
    The noise level $\sigma$ is conditioned using a time embedding module with linear layers.

    \item ResNet: we use a minimal ResNet with residual block~\citep{he2016deep}, with a $1 \times 1$ convolutional layer at the beginning and at the end, similar to~\citep{kamb2025an}. Each residual block contains a convolutional layer with $3 \times 3$ kernels at a channel dimension defined by \texttt{num\_channels}, followed by a ReLU nonlinearity. The noise level $\sigma$ is conditioned using sine-cosine positional embedding.    

    \item PatchMLP: a fully local MLP acting on patches. The MLP has 5 residual blocks, each containing two linear layers with hidden dimension of \texttt{hidden\_dim}, followed by a layer normalization and GELU activation. 
    The noise level $\sigma$ is conditioned using sine-cosine positional embedding. 
\end{itemize}
All convolutional layers use circular padding to ensure translation equivariance and they have approximately $4$ million parameters. 
Details of the architectures with various receptive fields are provided in~\cref{table:supp_architecture_details}.
    \begin{table}[t!]
        \caption{Details of neural network architectures with various receptive fields used in our experiments for images at $32 \times 32$ resolution.}
        \label{table:supp_architecture_details}
        \vspace*{-0.25cm}
        \begin{center}
        \begin{small}
        \begin{sc}
            \begin{tabular}{lccccc}
            \toprule
            & \multirow{2}{*}{\shortstack{Receptive field \\ (patch size)}} & \multicolumn{2}{c}{\multirow{2}{*}{Archi. hyper-parameters}} & \multirow{2}{*}{Num. parameters} \\
            & & & & \\
            \midrule
            % ---------------------------------------
            %         UNet2D
            % \multicolumn{4}{c}{UNet2D} \\
            UNet2D & & {\normalfont  \texttt{block\_out\_channels}} & {\normalfont  \texttt{kernel\_size}} & \\ 
            \midrule
            & $5$ & $(96, 224, 480)$     & $(3, 1, 1, 1)$--$1$ & 4.6M  \\
            & $7$ & $(64, 192, 448)$     & $(3, 1, 1, 1)$--$3$ & 5.1M \\
            & $9$ & $(96, 192, 288)$     & $(3, 1, 1, 1)$--$5$ & 4.3M \\
            & $11$ & $(64, 96, 160, 288)$     & $(3, 1, 1, 1, 3)$--$1$ & 4.3M \\
            % ---------------------------------------
            %         ResNet
            \midrule
            % \multicolumn{4}{c}{ResNet} \\
            ResNet & & {\normalfont  \texttt{num\_res\_blocks}} & {\normalfont  \texttt{num\_channels}} & \\ 
            \midrule
            & $5$ & $1$     & $640$ & 4.5M  \\
            & $7$ & $2$     & $448$ & 4.2M \\
            & $9$ & $3$     & $384$ & 4.5M \\
            & $11$ & $4$     & $328$ & 4.4M \\
            % ---------------------------------------
            %         MLP
            \midrule
            % \multicolumn{4}{c}{PatchMLP} \\
            PatchMLP & & {\normalfont  \texttt{hidden\_dim}} & {\normalfont  \texttt{num\_blocks}} & \\ 
            \midrule
            & $5$ & $7168$     & $5$ & 3.5M  \\
            & $7$ & $6144$     & $5$ & 3.7M \\
            & $9$ & $5120$     & $5$ & 4.3M \\
            & $11$ & $3072$     & $5$ & 4.0M \\
            \bottomrule
            \end{tabular}
        \end{sc}
        \end{small}
        \end{center}
        \vskip -0.1in
    \end{table} 

\subsection{Training procedure}\label{sec:supp_training_procedure}
All models are trained on a single NVIDIA A100 GPU with the following settings:
\begin{itemize}
    \item Optimizer: Adam optimizer~\citep{kingma2015adam}
    \item Learning rate: starting at $10^{-4}$ with cosine decay schedule and minimum learning rate at $10^{-6}$. 
    \item Number of epochs: $600$ for images at $32 \times 32$ resolution and $900$ for images at $64 \times 64$ resolution.
    \item Batch size: $256$
    \item Exponential moving average (EMA) with decay rate $0.99$ from the 1000-th training step and updates every $5$ steps. The EMA weights are used for evaluation.  
\end{itemize}
\subsection{Forward operators}
We consider the $3$ representative inverse problems as forward operators $A$:
\begin{itemize}
    \item Denoising: the forward operator is simply the identity $A = \Id$ and only the physics-agnostic estimator is applicable in this case.
    \item Inpainting: we consider the inpainting operator with a center square mask of size $15 \times 15$. The forward operator $A$ is therefore a diagonal matrix with $0$ on the masked pixels and $1$ elsewhere. 
    In this case, the pseudo-inverse is $BA^+  \!=\! A$: it simply removes noise inside the masked region and keeps the observed pixels unchanged.
    In our experiments, we consider the physics-aware estimator with $B = A^+ + \epsilon \Id$, where $\epsilon = 10^{-5}$ is a small regularization parameter. 
    This ensures that $B$ is full-rank and we can apply the constant-rank formula of the LE-MMSE estimator in~\cref{theorem:local_trans_estimator}. 
    As $\epsilon$ is very small, this estimator closely approximates the ideal physics-aware estimator with $B = A^+$, as discussed in~\cref{remark:rank_deficient_B}. 
    \item Deconvolution: we consider an isotropic Gaussian blur kernel with standard deviation $1.0$ with circular boundary conditions. The same kernel is used for all color channels. 
    In this case, we build the full matrix $A$ as a block-circulant matrix representing the convolution operation and $B$ is its pseudo-inverse, which is computed once, in double precision.  
\end{itemize}
\vfill
\subsection{Analytical formula implementation}
Implementing the analytical formulas of the E-MMSE~\cref{thm:equivariant_mmse} and the LE-MMSE~\cref{theorem:local_trans_estimator} estimators requires computing many distance terms between images or patches in the dataset $\D$ and the observed measurement $y$. We process by batch to avoid memory overflow.
For numerical stability and avoidance of overflow/underflow, the exponential terms are accumulated using an online log-sum-exp trick.
All theoretical estimators are computed in an \emph{exact} manner without any approximation, modular finite precision arithmetic. 
We use PyTorch~\citep{paszke2019pytorch} for implementation and all computations are performed in single precision (FP32) on a single A-100 GPU, unless otherwise specified.
All estimators are implemented, even the rank-deficient cases. 
Full implementation details are available at \url{\gitrepo}.

\pagebreak
\section{Additional numerical results}
\subsection{Comparison between trained neural networks and the analytical LE-MMSE estimator}\label{sec:supp_neural_vs_analytical}
We show in~\cref{fig:neural_vs_analytical_unet2d_patch_11,fig:neural_vs_analytical_resnet_patch_11,fig:neural_vs_analytical_patchmlp_patch_11} additional PSNR results between trained neural networks and the analytical LE-MMSE estimator for different architectures (UNet2D, ResNet, PatchMLP) on both training and test sets of various datasets. 
We recover a consistent conclusion across different settings: the trained neural networks closely approximate the analytical LE-MMSE estimator, with PSNR values exceeding $20$ in most cases and often reaching above $30$ dB.
% PSNR results for different architectures with patch size 11
\begin{figure*}[ht!]
    \centering
    \def\base{./images/neural_vs_analytical/localequivunet2dcondmodel_patch_11/plots}
    \begin{minipage}{0.95\linewidth}
        \includegraphics[width=\linewidth]{\base/FFHQ_images32x32_subset_10000_combined_psnr_grid.pdf}
        \subcaption{FFHQ-32}
    \end{minipage}
    \begin{minipage}{0.95\linewidth}
        \includegraphics[width=\linewidth]{\base/CIFAR10_subset_10000_combined_psnr_grid.pdf}
        \subcaption{CIFAR-10}
    \end{minipage}
    \begin{minipage}{0.95\linewidth}
        \includegraphics[width=\linewidth]{\base/FashionMNIST_subset_10000_combined_psnr_grid.pdf}
        \subcaption{Fashion-MNIST}
    \end{minipage}
    \caption{Additional PSNR between trained UNet2D and the analytical formula of the LE-MMSE for different inverse problems on both training and test sets of various datasets. 
    The patch size is $P = 11 \times 11$.
    Left: physics-agnostic estimator with $B\!=\!\Id$. Right: physics-aware estimator with $B\!=\!A^+$.
    We recover the same conclusions as in~\cref{fig:neural_vs_analytical_unet2d_patch_5}.
    }
    \label{fig:neural_vs_analytical_unet2d_patch_11} 
\end{figure*}

\begin{figure*}[h!]
    \centering
    \def\base{./images/neural_vs_analytical/minimalresnet_patch_11/plots}
    \begin{minipage}{\linewidth}
        \includegraphics[width=\linewidth]{\base/FFHQ_images32x32_subset_10000_combined_psnr_grid.pdf}
        \subcaption{FFHQ-32}
    \end{minipage}
    \begin{minipage}{\linewidth}
        \includegraphics[width=\linewidth]{\base/CIFAR10_subset_10000_combined_psnr_grid.pdf}
        \subcaption{CIFAR-10}
    \end{minipage}
    \begin{minipage}{\linewidth}
        \includegraphics[width=\linewidth]{\base/FashionMNIST_subset_10000_combined_psnr_grid.pdf}
        \subcaption{Fashion-MNIST}
    \end{minipage}
    \caption{PSNR between trained ResNet and the analytical formula of the LE-MMSE for different inverse problems on both training and test sets of various datasets. 
    The patch size is $P = 11 \times 11$.
    Left: physics-agnostic estimator with $B\!=\!\Id$. Right: physics-aware estimator with $B\!=\!A^+$.
    We recover the same conclusions accross architectures and settings.
    }
    \label{fig:neural_vs_analytical_resnet_patch_11} 
\end{figure*}

\begin{figure*}[h!]
    \centering
    \def\base{./images/neural_vs_analytical/patchmlp_patch_11/plots}
    \begin{minipage}{\linewidth}
        \includegraphics[width=\linewidth]{\base/FFHQ_images32x32_subset_10000_combined_psnr_grid.pdf}
        \subcaption{FFHQ-32}
    \end{minipage}
    \begin{minipage}{\linewidth}
        \includegraphics[width=\linewidth]{\base/CIFAR10_subset_10000_combined_psnr_grid.pdf}
        \subcaption{CIFAR-10}
    \end{minipage}
    \begin{minipage}{\linewidth}
        \includegraphics[width=\linewidth]{\base/FashionMNIST_subset_10000_combined_psnr_grid.pdf}
        \subcaption{Fashion-MNIST}
    \end{minipage}
    \caption{PSNR between trained PatchMLP neural network and the analytical formula of the LE-MMSE for different inverse problems on both training and test sets of various datasets.      
    The patch size is $P = 11 \times 11$.
    We recover the same conclusions as in~\cref{fig:neural_vs_analytical_unet2d_patch_5,fig:neural_vs_analytical_unet2d_patch_11,fig:neural_vs_analytical_resnet_patch_11}. An exception is observed for the deconvolution task with physics-aware models (rightmost column). Here, the noise is amplified by the inversion of the blurring operator and the local PatchMLP architecture struggles to accurately reconstruct fine details, leading to a lower PSNR comparing to CNNs. We hypothesize that CNNs have other inductive biases that are more suited to handle such challenging tasks.    
    }
    \label{fig:neural_vs_analytical_patchmlp_patch_11} 
\end{figure*}
\null
\pagebreak

% # Qualitative comparison for different patch sizes
\subsection{Addition qualitative comparison}
We provide additional qualitative comparisons between the analytical LE-MMSE estimator and trained neural networks (UNet2D, ResNet, PatchMLP) for different patch sizes in~\cref{fig:qualitative_neural_vs_analytical_patch_5,fig:qualitative_neural_vs_analytical_patch_7,fig:qualitative_neural_vs_analytical_patch_9,fig:qualitative_neural_vs_analytical_patch_11} on the FFHQ, CIFAR10 and FashionMNIST datasets.
Overall, the neural networks closely match the analytical solution on both training and test sets across all datasets and inverse problems, confirming our theoretical findings. 
\begin{figure*}[ht!]
        \centering
        \vskip -0.1in
        \includegraphics[trim=35mm 0 0 0, clip,width=\textwidth]{./tex_figure/fig_qualitative_neural_vs_analytical_patch_5_standalone.pdf}
        \vskip -0.1in
        \caption{Qualitative comparison. 
        The patch size is $P = 5 \times 5$ and $B\!=\!\Id$.
        The noise level is $\sigma = 0.05,0.2,0.8$ from left to right for each column.
        The neural networks (UNet2D, ResNet, and PatchMLP) closely match the analytical LE-MMSE on both training and test sets across all datasets (FFHQ, CIFAR10, and FashionMNIST) and inverse problems, confirming our theoretical findings. 
        }
        \label{fig:qualitative_neural_vs_analytical_patch_5} 
        \vskip -0.15in
\end{figure*}

\begin{figure*}[ht!]
        \centering
        \vskip -0.1in
        \includegraphics[trim=35mm 0 0 0, clip,width=\textwidth]{./tex_figure/fig_qualitative_neural_vs_analytical_patch_7_standalone.pdf}
        \vskip -0.15in
        \caption{Qualitative comparison between the analytical formula LE-MMSE, UNet2D, ResNet, and PatchMLP, with $B\!=\!\Id$ on FFHQ, CIFAR10 and FashionMNIST. 
        The patch size is $P=7 \times 7$.
        The noise level is $\sigma = 0.05,0.2,0.8$ from left to right for each column.
        The neural networks closely match the analytical solution on both training and test sets across all datasets and inverse problems, confirming our theoretical findings. 
        Yet, some discrepancies can be observed, especially at low noise levels on the test set, which we attribute to generalization issues discussed in~\cref{sec:neural_generalization}.
        }
        \label{fig:qualitative_neural_vs_analytical_patch_7} 
        \vskip -0.15in
\end{figure*}

\begin{figure*}[ht!]
        \centering
        \vskip -0.1in
        \includegraphics[trim=35mm 0 0 0, clip,width=\textwidth]{./tex_figure/fig_qualitative_neural_vs_analytical_patch_9_standalone.pdf}
        \vskip -0.15in
    \caption{Qualitative comparison between the analytical formula LE-MMSE, UNet2D, ResNet, and PatchMLP, with $B\!=\!\Id$ on FFHQ, CIFAR10 and FashionMNIST. 
    The patch size is $P=9$.
    The noise level is $\sigma = 0.05,0.2,0.8$ from left to right for each column.
    The neural networks closely match the analytical solution on both training and test sets across all datasets and inverse problems, confirming our theoretical findings. 
    Yet, some discrepancies can be observed, especially at low noise levels on the test set, which we attribute to generalization issues discussed in~\cref{sec:neural_generalization}.
    }
	\label{fig:qualitative_neural_vs_analytical_patch_9} 
    \vskip -0.15in
\end{figure*}

\begin{figure*}[ht!]
        \centering
        \vskip -0.1in
        \includegraphics[trim=35mm 0 0 0, clip,width=\textwidth]{./tex_figure/fig_qualitative_neural_vs_analytical_patch_11_standalone.pdf}
        \vskip -0.15in
    \caption{Qualitative comparison between the analytical formula LE-MMSE, UNet2D, ResNet, and PatchMLP, with $B\!=\!\Id$ on FFHQ, CIFAR10 and FashionMNIST. 
    The patch size is $P=11$.
    The noise level is $\sigma = 0.05,0.2,0.8$ from left to right for each column.
    The neural networks closely match the analytical solution on both training and test sets across all datasets and inverse problems, confirming our theoretical findings. 
    Yet, some discrepancies can be observed, especially at low noise levels on the test set, which we attribute to generalization issues discussed in~\cref{sec:neural_generalization}.
    }
	\label{fig:qualitative_neural_vs_analytical_patch_11} 
    \vskip -0.15in
\end{figure*}

\null
\pagebreak
\null
\pagebreak
\null
\pagebreak
\subsection{Influence of the patch size (receptive field)}\label{sec:supp_patch_size}
\paragraph{Density of the patch distribution}
We analyze here the influence of the patch size on the density of the patch distribution in FFHQ-32 dataset. 
We compute the negative-log-density of patches as a function of the patch size, for points on either the training set or the test set.  
The patch density is exactly the denominator term in the analytical LE-MMSE formula~\cref{eq:loc_equiv_formula}.
The results are shown in~\cref{fig:supp_patch_size_vs_density}.
\begin{figure}[ht!]
    \centering
    \vskip -0.1in
    \includegraphics[width=\linewidth]{./images/analytical_density_vs_patch/denoising_FFHQ_images32x32_subset_10000_patches_5-7-9-11-15-19_train.pdf}
    \includegraphics[width=\linewidth]{./images/analytical_density_vs_patch/denoising_FFHQ_images32x32_subset_10000_patches_5-7-9-11-15-19_test.pdf}
    \vskip -0.1in
    \caption{The negative-log-density of patches in FFHQ-32 training set (top) and test set (bottom) as a function of the patch size.
    As the patch size increases, the density of patches decreases significantly, indicating a sparser coverage of the patch space by the dataset. 
    In the training set, the negative-log-density is relatively lower (around $10^2$) than in the test set (around $10^4 $ for small noise levels), indicating a better coverage of the patch space by the training set. 
    This observation helps explain the drop in PSNR between trained neural networks and the analytical LE-MMSE estimator in the test set for low noise levels.
    }
    \label{fig:supp_patch_size_vs_density}
\end{figure}

\null 
\pagebreak
\paragraph{Patch size influence}
In~\cref{fig:supp_neural_analytic_vs_patch_size}, we show the PSNR between the trained UNet2D and the analytical LE-MMSE estimator for different patch sizes on both training and test sets of FFHQ-32 dataset. 
For low noise levels and in the test set, the PSNR decreases as the patch size increases, which we attribute to the sparser coverage of the patch space by the dataset for larger patches, as shown in~\cref{fig:supp_patch_size_vs_density}. 

\begin{figure}[ht!]
    \centering
    \vskip -0.1in
    \includegraphics[width=.75\linewidth]{./images/neural_vs_analytical/localequivunet2dcondmodel_patch_5_7_9_11/plots/FFHQ_images32x32_subset_10000_patch_grid_nva_only.pdf}
    \vskip -0.1in
    \caption{PSNR between trained UNet2D and the analytical formula of the LE-MMSE versus the patch size, on the FFHQ-32 dataset.
    Top: on the training set. Bottom: on the test set.
    }
    \label{fig:supp_neural_analytic_vs_patch_size}
\end{figure}

In~\cref{fig:supp_analytic_vs_gt_patch_size}, we show the PSNR between the analytical LE-MMSE estimator and the ground truth as a function of the patch size on both training and test sets of FFHQ-32 dataset.
We observe that -- on the test set -- smaller patch size are preferable for low noise levels, while larger patch sizes yield better performance for high noise levels.
The behavior on the training set is different: for low noise levels, the analytical formula copy-pastes exactly the right patches, explaining the blow up of PSNR at the origin.

\begin{figure}[ht!]
    \centering
    \vskip -0.1in
    \includegraphics[width=.75\linewidth]{./images/rec_perf/localequivunet2dcondmodel_patch_5_7_9_11/plots/FFHQ_images32x32_subset_10000_patch_grid_analytical_vs_gt.pdf}
    \vskip -0.1in
	\caption{PSNR between the analytical formula of the LE-MMSE and the ground truth versus the patch size, on the FFHQ-32 dataset. 
    Top: on the training set. Bottom: on the test set.
    }
    \label{fig:supp_analytic_vs_gt_patch_size}
\end{figure}

% \null
% \pagebreak

\subsection{Mass concentration}\label{sec:supp_mass_concentration}
The LE-MMSE estimator~\cref{theorem:local_trans_estimator} is a weighted average of the central pixel values of patches in the dataset. 
To understand the behavior of the estimator, we analyze how many patches contribute significantly to the estimate at each pixel location.
In~\cref{fig:supp_mass_concentration}, we show the number of patches contributing to $99\%$ of the mass of the LE-MMSE estimator as a function of the noise level $\sigma$.
We observe that for low noise levels, the estimator concentrates its mass on fewer patches (even nearest neighbors).
As the noise level increases, the number of contributing patches increases significantly, with a critical value of $\sigma$ where the number of patches starts to increase rapidly.
This behavior shows that although the MMSE estimator is typically associated with averaging, it can behave like a nearest-neighbor estimator when the noise is small, due to the strong concentration of mass on a very limited subset of patches.
We also observe a clear difference between physics-agnostic and physics-informed settings for deconvolution, where the latter has a significantly higher number of contributing patches due to the amplification of noise by the pseudo-inverse of the blurring operator.
\begin{figure}[ht!]
    \centering
    \vskip -0.1in
    \includegraphics[width=0.6\columnwidth]{./images/patch_works/patch_works_effective_samples_FFHQ_images32x32_p5.pdf}
    \vskip -0.1in
    \caption{
    For low noise levels, the LE-MMSE estimator concentrates its mass on fewer patches (even nearest neighbors).
    Here we show the median and IQR (over all pixels of $50$ samples on the test set, for each $\sigma$) of the number of patches contributing to $99\%$ of the mass of the LE-MMSE estimator. 
    There is a significant increase in number of contributed patches as the noise level $\sigma$ increases, 
    and a critical value of $\sigma$ where the number of patches starts to increase rapidly.
    The patch size is $P = 5 \times 5$, on FFHQ-32.
    Note that due to computational constraints, we only keep the top $10^4$ nearest patches (over $32 \times 32 \times 10^4 \approx 10^7$ patches) when computing mass concentration, but the estimator is still exact as we accumulate all the mass in an online manner.
    }
    \label{fig:supp_mass_concentration}
    \vskip -0.1in
\end{figure}
We also visualize in~\cref{fig:supp_mass_concentration_visual,fig:supp_mass_concentration_visual_inform} the $\log_{10}$ of number of patches contributing to $99\%$ of the mass of the LE-MMSE estimator at each pixel location for different inverse problems on FFHQ-32 dataset.
We observe that for low noise levels, the nearest patch (or a very small number of patches) contributes $99\%$ of the mass at each pixel location. 
\begin{figure}[ht!]
    \centering
    \vskip -0.1in
    \def\root{./images/patch_works_visuals/visual_FFHQ_images32x32}
    \def\size{0.95\linewidth}
    \begin{minipage}{\size}
        \includegraphics[width=\linewidth]{\root_denoising_agnostic_img0_p5.pdf}
        \subcaption{Denoising}
    \end{minipage}
    \begin{minipage}{\size}
        \includegraphics[width=\linewidth]{\root_inpainting_center_15_agnostic_img1_p5.pdf}
        \subcaption{Inpainting (physic-agnostic)}
    \end{minipage}
    \begin{minipage}{\size}
        \includegraphics[width=\linewidth]{\root_convolution_gaussian_1.0_agnostic_img2_p5.pdf}
        \subcaption{Deconvolution (physic-agnostic)}
    \end{minipage}
    % \begin{minipage}{\size}
    %     \includegraphics[width=\linewidth]{\root_convolution_gaussian_1.0_inform_img2_p5.pdf}
    %     \subcaption{Deconvolution (physic-informed)}
    % \end{minipage}
    % \begin{minipage}{\size}
    %     \includegraphics[width=\linewidth]{\root_inpainting_center_15_inform_img1_p5.pdf}
    %     \subcaption{Inpainting (physic-informed)}
    % \end{minipage}
    \caption{Visualization of the number of patches (in $\log_{10}$ scale) contributing to $99\%$ of the mass of the physic-agnostic LE-MMSE estimator at each pixel location for different inverse problems on FFHQ-32 dataset.
    }\label{fig:supp_mass_concentration_visual}
    \vskip -0.1in
\end{figure}
\begin{figure}[ht!]
    \centering
    \vskip -0.1in
    \def\root{./images/patch_works_visuals/visual_FFHQ_images32x32}
    \def\size{0.95\linewidth}
    \begin{minipage}{\size}
        \includegraphics[width=\linewidth]{\root_inpainting_center_15_inform_img1_p5.pdf}
        \subcaption{Inpainting (physic-informed)}
    \end{minipage}
    \begin{minipage}{\size}
        \includegraphics[width=\linewidth]{\root_convolution_gaussian_1.0_inform_img2_p5.pdf}
        \subcaption{Deconvolution (physic-informed)}
    \end{minipage}
    \caption{Visualization of the number of patches (in $\log_{10}$ scale) contributing to $99\%$ of the mass of the physic-inform LE-MMSE estimator at each pixel location for different inverse problems on FFHQ-32 dataset.
    For deconvolution, the number of contributing patches is significantly higher than for the physic-inform case, due to the amplification of noise by the pseudo-inverse of the blurring operator.
    }
    \vskip -0.1in
    \label{fig:supp_mass_concentration_visual_inform}
\end{figure}

\null
\pagebreak
\null
\pagebreak
\subsection{LE-MMSE is a patchwork}\label{sec:supp_patch_works}
From the LE-MMSE formula~\cref{theorem:local_trans_estimator}, we see that the estimate at each pixel location is a weighted average of the central pixel values of patches in the dataset.
In fact, the estimator often concentrates its mass on very few patches, as shown in~\cref{sec:supp_mass_concentration}. 
Moreover, contiguous pixels may come from the central pixels of patches of the same image in the dataset, leading to a patchwork behavior. 

We visualize in~\cref{fig:supp_patch_index_ffhq,fig:supp_patch_index_FashionMNIST} this behavior for different inverse problems on FFHQ-32 and FashionMNIST datasets. 
More precisely, at each pixel location, the source index is the index of the image in the dataset that contains the patch whose central pixel contributes at least $50\%$ of the mass of the LE-MMSE estimator at that pixel location.
We observe that, large contiguous regions of the estimate come from the same source image in the dataset, which we can interpret as the estimator to be a patchwork of training patches.
As the noise level increases, more pixels become white, meaning that the corresponding pixel value is not mostly due to a single patch: the ``patchwork effect'' diminishes, as more patches contribute to the estimate at each pixel location. 
\begin{figure*}[ht!]
    \centering
    % \vskip -0.1in
    \includegraphics[trim=10mm 0 0 0,clip,width=0.9\linewidth]{./tex_figure/fig_patch_index_patch_11_standalone.pdf}
    \vskip -0.1in
    \caption{Illustration of the patchwork behavior of the LE-MMSE estimator (with $B\!=\!\Id$) on FFHQ-32 dataset for different inverse problems and noise levels. We use a patch size of $P = 11 \times 11$. 
    We observe contiguous regions coming from the same source image in the dataset for low noise levels.
    White pixels correspond to locations where no patch contributes more than $50\%$ of the mass.
    }
    \label{fig:supp_patch_index_ffhq}
\end{figure*}
\begin{figure*}[ht!]
    \centering
    \includegraphics[trim=10mm 0 0 0,clip,width=0.9\linewidth]{./tex_figure/fig_patch_index_patch_11_fashion_mnist_standalone.pdf}
    \vskip -0.1in
    \caption{Similar illustration as in~\cref{fig:supp_patch_index_ffhq} but on FashionMNIST dataset.}
    \label{fig:supp_patch_index_FashionMNIST}
\end{figure*}

Note that we use a patch size of $P = 11 \times 11$ to better visualize the patchwork behavior. 
This choice results in slightly weaker reconstruction performance, as discussed in~\cref{sec:supp_patch_size}.

\null
\pagebreak
\subsection{Out-of-distribution data}\label{sec:supp_ood}
We provide additional results on out-of-distribution (OOD) data using UNet2D architecture with receptive field of size $5 \times 5$.
The models and the formula are trained / evaluated on $\D=$ FFHQ-32. They are then evaluated on OOD dataset $\D'=$ CIFAR10. 
We show in~\cref{fig:supp_ood_ffhq_cifar10} the PSNR between trained UNet2D and the analytical LE-MMSE estimator.
For large noise levels, as the Gaussians overlap significantly even on OOD data, the value of $-\log p(y)$ is low and the trained neural network closely matches the analytical LE-MMSE estimator with PSNR values above $25$ dB. 
For low noise levels, the PSNR is about $3$ dB lower than in the in-distribution case shown in~\cref{fig:neural_vs_analytical_unet2d_patch_5}, which we attribute to the low-density of the measurement density $p(y)$, as discussed in~\cref{sec:neural_generalization} and can be seen in~\cref{fig:supp_ood_ffhq_cifar10}.
\begin{figure}[ht!]
    \centering
    \vskip -0.1in
    \includegraphics[width=0.65\columnwidth]{./images/ood/localequivunet2dcondmodel_patch_5/ood_psnr_vs_sigma_FFHQ_images32x32.pdf} \\
    \includegraphics[width=0.65\columnwidth]{./images/ood/localequivunet2dcondmodel_patch_5/ood_density_vs_sigma_FFHQ_images32x32.pdf}
    \vskip -0.1in
    \caption{Comparison of UNet2D and the analytical LE-MMSE estimator on OOD dataset CIFAR10 when both are trained on FFHQ-32.
    Median and IQR using $50$ images per $\sigma$, $P = 5\times 5$ and $B\!=\!\Id$.
    On top: PSNR between UNet2D and the analytical LE-MMSE estimator.
    On bottom: negative-log-density of measurements $y$ in CIFAR10 under the measurement distribution induced by FFHQ-32. 
    }
    \label{fig:supp_ood_ffhq_cifar10}
\end{figure}

\null
\pagebreak
\subsection{Dataset size influence}\label{sec:supp_experiment_dataset_size}
We analyze here the influence of the dataset size on the alignment between trained neural networks and the analytical LE-MMSE estimator.
We train UNet2D models with receptive field of size $P = 5 \times 5$ and $B\!=\!\Id$ on various dataset sizes from $10^3$ to $5 \times 10^4$ images from FFHQ-32.
The PSNR between trained UNet2D and the analytical LE-MMSE estimator is reported in~\cref{fig:dataset_size_influence}.
We observe that the dataset size has limited influence on the alignment between trained neural networks and the analytical LE-MMSE formula, with a slight improvement when increasing the dataset size. 
\begin{figure}[ht!]
    \centering
    \vskip -0.1in
    \includegraphics[width=0.75\columnwidth]{./images/neural_vs_analytical/localequivunet2dcondmodel_patch_5/plots/FFHQ_images32x32full70k_sizes_1000_5000_10000_20000_30000_40000_50000_patch_5_psnr_vs_sigma_train_test.pdf}
    \vskip -0.1in
    \caption{
        Dataset size has limited influence on the alignment between neural networks and the LE-MMSE formula.
        Median and IQR using $50$ images per $\sigma$, $P = 5\times 5$ and $B\!=\!\Id$.
    }
    \label{fig:dataset_size_influence}
    \vskip -0.1in
\end{figure}

\null
\pagebreak
\subsection{Results on $3 \times 64 \times 64$ images}\label{sec:supp_experiment_64x64}
We provide additional results on images at $3 \times 64 \times 64$ resolution using UNet2D architecture with receptive field of size $11 \times 11$.
The models are trained on $10^4$ images from FFHQ downscaled to $64 \times 64$, with the same training procedure as in~\cref{sec:supp_training_procedure}.
    \begin{table}[ht!]
        \caption{Details of neural network architectures with various receptive fields used in our experiments for images at $3 \times 64 \times 64$ resolution.}
        \label{table:supp_architecture_unet2d_64}
        \vspace*{-0.25cm}
        \begin{center}
        \begin{small}
        \begin{sc}
            \begin{tabular}{lccccc}
            \toprule
            & \multirow{2}{*}{\shortstack{Receptive field \\ (patch size)}} & \multicolumn{2}{c}{\multirow{2}{*}{Archi. hyper-parameters}} & \multirow{2}{*}{Num. parameters} \\
            & & & & \\
            \midrule
            % ---------------------------------------
            %         UNet2D
            % \multicolumn{4}{c}{UNet2D} \\
            UNet2D & & {\normalfont  \texttt{block\_out\_channels}} & {\normalfont  \texttt{kernel\_size}} & \\ 
            \midrule
            & $11$ & $(64, 128, 256, 512)$     & $(3, 1, 1, 1, 3)$--$1$ & 13.3M \\
            \bottomrule
            \end{tabular}
        \end{sc}
        \end{small}
        \end{center}
        \vskip -0.1in
    \end{table}

\begin{figure*}[ht!]
    \centering
    % \vskip -0.1in
    \includegraphics[trim=16mm 0 0 0,clip,width=\linewidth]{./tex_figure/fig_qualitative_neural_vs_analytical_patch_11_64x64_more_standalone.pdf}
    \caption{Additional qualitative comparison between UNet2D and the analytical LE-MMSE estimator on FFHQ-64 across tasks.}
    \label{fig:supp_additional_qualitative_64x64}
\end{figure*}
% ########################################################################################################



\end{document}